\documentclass{proc-l}
\usepackage[utf8]{inputenc}

\newtheorem{theorem}{Theorem}[section]
\newtheorem{lemma}[theorem]{Lemma}

\theoremstyle{definition}
\newtheorem{definition}[theorem]{Definition}
\newtheorem{example}[theorem]{Example}
\newtheorem{xca}[theorem]{Exercise}

\theoremstyle{remark}
\newtheorem{remark}[theorem]{Remark}

\numberwithin{equation}{section}



\title{math-homework}
\author{Johnicholas Hines}
\date{July 2021}

\begin{document}

\maketitle

\section{Products are products and substituting an isomorphic object leads to an equivalent category}

%Consider the rule
%\[
%  \mathrm{Pow} \colon \mathrm{Set} \to \mathrm{Set}
%\]
%which associates a set \(S\) to its power set
%\[
%  \mathrm{Pow}(S) := \{ X \mid X \subset S \}
%\]
%and which associates a function \(f \colon S \to T\)
%the inverse image
%\begin{align*}
%  \mathrm{Pow}(f) \colon \mathrm{Pow}(T) &\to \mathrm{Pow}(S) \\
%  Y & \mapsto f^{-1}(Y).
%\end{align*}
%We check this is a contravariant functor...



This is in response to this prompt:

\begin{quote}
Consider the category of sets. Let X and Y be two sets. The Cartesian product
\[
X \times Y = \{(x,y) | x \in X, y \in Y\}
\]
is another set. Moreover this has the following property: given any other set \(Z\) and two functions \(f: Z \to X\) and \(g: Z \to Y\), there is a unique function \(h: Z \to X \times Y\) such that \(h; pr X = f\) and \(h ; pr Y = g\). Here, \(pr X: X \times Y \to X\) and \(pr Y: X \times Y \to Y\) are the projection maps. (One might say that some diagram commutes…) Compare this with the notion of a categorical product. 

There are two things to check: that there is a map, and that it is unique. It is somehow so tautological that the proof might be hard.
\end{quote}

(Regarding notation, remember we are using the semicolon `followed by' operator \(f; g = g \circ f\) for reversed function composition, and juxtaposition for function application; \(f x\) means \(f\) applied to \(x\).)

\subsection{products are products}

When I was a kid, there was a `puzzle' or `paradox', possibly inspired by superhero stories, asking what would happen, if an unstoppable force met an immovable obstacle.\footnote{https://en.wikipedia.org/wiki/Irresistible\_force\_paradox}
I don't remember whether it was my older cousin who told it to me, but it might have been, and let's say it was.
It was a bit cruel (kids are often a bit cruel to each other), or at least, it felt a bit cruel to me.
When my cousin asked the question,
they subtly implied that the answer was either one or the other (as it turns out, a false dichotomy),
and that the question was a matter of taste. It is not a matter of taste. In contrast, ``Who is your favorite superhero?'' is a question that \emph{is} a matter of taste.
Then whichever side I picked, my cousin would talk about the opposite side's guarantee,
building up their specified superpower by using emphasis and superlatives,
saying things like ``but this force is \emph{absolutely unstoppable}'',
until I felt bullied into switching my answer.

For me, the key to eventually understanding what was going on was to stop narrowly trying to work forward from the stipulated qualities, or narrowly trying to work backward from one or the other (presumed) possibly-correct answers, but instead to relax and think about a large grid of "when this bumps that, what happens?", possibly with little xkcd-style illustrations\footnote{something like https://xkcd.com/2043/ or https://xkcd.com/1890/} in each cell.


I think of this as `naturalistic' thinking, your mind having a gentle and wide grip rather than a narrow tight focus. The idea is that we are collecting flower-theorems from this entire field of flower-theorems, all of which are valuable. The naturalist-mathematician is interested in building a catalog of all the flower-theorems, not breaking down anything or building up to anything in particular.

The quality of `unstoppableness' is like circling a row and saying that everywhere in this row, the bumper overcomes the bumpee.
The quality of `irresistablenesss' is like circling one column and saying that everywhere in this column, the bumpee stops the bumper.
The grid and the circling can resolve the paradox.
In this case, the resolution is that the puzzle setter has contradicted themselves.

There's a similar puzzle\footnote{https://en.wikipedia.org/wiki/Teumessian\_fox} regarding a dog, Laelaps, who is guaranteed to succeed in catching everything it hunts and a fox who is guaranteed to evade any pursuer,
which I mention because the resolution to the puzzle is different in this case: the hunt turns out to be nonterminating.

Turing's halting problem and Cantor's proof of the uncountability of real numbers are
more arguments which I think of as flowing from this `think about the whole grid of...'
heuristic.

In group theory, the identity satisfies the law that for all \(x\), \(x * e = e * x = e\).
Suppose that there was another element \(e_2\) which also satisfied a law that for all \(x\), \(x * e_2 = e_2 * x = e_2\).
Satisfying this property means there is a whole row (the \(e_2\) row) and a whole column (the \(e_2\) column) filled with \(e_2\) in the multiplication table of the group.
Of course, there is also a whole row (the \(e_1\) row) and a whole column (the \(e_1\) column) filled with \(e_1\).
What happens when they intersect?

Here again the resolution is different, the four products \(e\{1,2\} * e\{1,2\}\) can all be equal to \(e_1\) and also all be equal to \(e_2\), if \(e_1\) is equal to \(e_2\).
\[
e_1 = \{e_1, e_2\} * \{e_1,e_2\} = e_2
\]

Note that we are using braces and commas here to talk about a bunch of alternatives\footnote{https://www.linuxjournal.com/content/bash-brace-expansion}, for example `\{a,b\} hello \{2,3\}' expands to
the four strings `a hello 2', `a hello 3', `b hello 2', `b hello 3'. In this case, all four of the expansions of \(\{e_1, e_2\} * \{e_1,e_2\}\) are equal, but sometimes you, the reader, may need to use some interpretation as to which expansions are intended.

In category theory, an initial object, 0,  satisfies the law that for all objects \(x\) there exists a unique map \(0 \to x\). What if there were two initial objects, \(0_1\) and \(0_2\)?
Then we could draw a grid of all the \(hom(x, y)\), and look at the rows and columns involving \(0_1\) and \(0_2]\).
The whole row of \(0_1\) is filled with singleton \(hom(0_1, y)\) sets,
and the whole row of \(0_2\) is filled with singleton \(hom(0_2, y)\) sets.
Looking down the columns, we don't know much - except of course where they intersect the rows.

For all four \(\{0_1,0_2\} \to \{0_1,0_2\}\) there is a unique map of that type,
and we know some \(0_1\to0_1\) and \(0_2\to0_2\) maps, the identity maps, already.
So \(0_1\to 0_2\) composed with \(0_2 \to 0_1\) must yield the identity map \(\operatorname{id} 0_1 : 0_1 \to 0_1\),
and \(0_2 \to 0_1\) composed with \(0_1 \to 0_2\) must yield the identity map \(\operatorname{id} 0_2 : 0_2 \to 0_2\).
That is, we use the two object's guarantees (their posited superpowers) to obtain an isomorphism between them.

The resolution of the argument regarding cartesian product and categorical product is close to the group-identities-must-be-equal and the initial-objects-must-be-equal
argument: we end up with the cartesian product and categorical product being isomorphic.

During the course of the argument, we need to hold the cartesian product and categorical product as
if they are distinct. But the similarity of the concepts means that its easy to be confused,
both as proof-writer and proof-reader, and (due to the similarity of the concepts) they also have 
similar names, which also makes it confusing.

\begin{definition}
Given \(X, Y\) sets, 
\(
\operatorname{Decartes} X Y := \{(x,y) | x \in X, y \in Y\}
\)
\end{definition}

\begin{theorem}[the universal property of the cartesian product]
Given any other set Z and two functions \(f: Z \to X\) and \(g: Z \to Y\),
there is a unique function \(h: Z \to \operatorname{Decartes} X Y\)
such that \(h; \operatorname{pr} X = f\) and \(h; \operatorname{pr} Y = g\).
Here \(\operatorname{pr} X: \operatorname{Decartes} X Y \to X\) and \(\operatorname{pr} Y: \operatorname{Decartes} X Y \to Y\) are the projection maps.
\end{theorem}


%Definition of Categorical Product (reworded from Wikipedia):
\begin{definition}

Fix a category \(C\). Let \(X_1\) and \(X_2\) be objects of C. A product of \(X_1\) and \(X_2\) is an object \(\operatorname{Cross} X_1 X_2\),
with a pair of arrows \(p_1: \operatorname{Cross} X_1 X_2 \to X_1\), \(p_2: \operatorname{Cross} X_1 X_2 \to X_2\)
such that
for every object Y and every pair of arrows \(f_1: Y \to X_1\), \(f_2: Y \to X_2\),
there exists a unique morphism \(f: Y \to \operatorname{Cross} X_1 X_2\) such that
\(f; p_1 = f_1\)
and 
\(f; p_2 = f_2\)
\end{definition}

\begin{theorem}
The Categorical Product is isomorphic to the Cartesian Product in Set.
\end{theorem}


\begin{proof}

We have \(X, Y\) sets. We want to show that \(\operatorname{Decartes} X Y\) is isomorphic to \(\operatorname{Cross} X Y\).

We take
\(\operatorname{Decartes} X Y\),
the cartesian projection \(\operatorname{decartesP} 1: \operatorname{Decartes} X Y \to X\) and
the cartesian projection \(\operatorname{decartesP} 2: \operatorname{Decartes} X Y \to Y\)
and use them with \(\operatorname{Cross} X Y\)'s guarantee to get
a unique arrow \(dc: \operatorname{Decartes} X Y \to \operatorname{Cross} X Y\)
such that
\[
dc; \operatorname{crossP} \{1,2\} = \operatorname{decartesP} \{1,2\}
\]

\begin{quote}
Note that here it is not that all four brace expansions are equal to one another, it's that corresponding expansions on one side are equal to parallel expansions on the other side. This is ambiguous notation, but sometimes ambiguous terse notation is better than precise verbose notation.
\end{quote}

We take 
\(\operatorname{Cross} X Y\),
the cross projection \(\operatorname{crossP} 1: \operatorname{Cross} X Y \to X\) and
the cross projection \(\operatorname{crossP} 2: \operatorname{Cross} X Y \to Y\)
and use them with \(\operatorname{Decartes} X Y\)'s guarantee to get
a unique arrow \(cd: \operatorname{Cross} X Y \to \operatorname{Decartes} X Y\)
such that
\[
cd; \operatorname{decartesP} \{1,2\} = \operatorname{crossP} \{1,2\}
\]

We take \(\operatorname{Decartes} X Y\), \(\operatorname{decartesP} 1\) and \(\operatorname{decartesP} 2\),
and use them with \(\operatorname{Decartes} X Y\)'s guarantee to get
a unique arrow \(dd: \operatorname{Decartes} X Y \to \operatorname{Decartes} X Y\)
such that
\[
dd; \operatorname{decartesP} \{1,2\} = \operatorname{decartesP} \{1,2\}
\]

We know another arrow, \(\operatorname{id} (\operatorname{Decartes} X Y)\), which has those properties but \(dd\) was guaranteed to be unique and so \(dd = \operatorname{id}\).

Consider \(dc; cd\).
\begin{align*}
dc; cd; \operatorname{decartesP} \{1,2\} & = dc; \operatorname{crossP} \{1,2\} \\
& = \operatorname{decartesP} \{1,2\} \\
\end{align*}

We know another arrow, \(dd = \operatorname{id}\) which has those properties, 
but it was guaranteed to be unique
and so \(dc; cd = \operatorname{id}\).

By a symmetrical argument, \(cd; dc = \operatorname{id}\). Therefore \(\operatorname{Decartes} X Y\) is isomorphic to \(\operatorname{Cross} X Y\).
\end{proof}

\subsection{substituting isomorphic objects leads to equivalent categories}

Before spending time thinking about it, I worried that substituting isomorphic objects for each other
might be `visible' in some sense via arrow composition maybe distant from the isomorphic objects. This is
an argument that we are licensed by an isomorphism to do `surgery' on the set of objects of a category,
collapsing pairs of isomorphic objects into one another.
\marginpar{maybe related to skeletal categories}

Given a category \(X\), and a pair of purportedly distinct objects in it named \(Decartes\) and \(Cross\),
and arrows \(dc: Decartes \to Cross\) and \(cd: Cross \to Decartes\), 
such that \(dc; cd = \operatorname{id}\) and \(cd; dc = \operatorname{id}\), let's try to make two new categories:
%1.
\(X \setminus Decartes\), which has almost all the objects and arrows in it that \(X\) had,
but doesn't have \(Decartes\) or any arrows that start or end at \(Decartes\).
2. \(X \setminus Cross\), which has almost all the objects and arrows in it that \(X\) had,
but doesn't have \(Cross\) or any arrows that start or end at \(Cross\).

Can we define a functor from \(X\) to \(X \setminus Decartes\)?
Yes, everything stays where it is, except for the things that got dropped need to be fixed up.
\(Decartes\) goes to \(Cross\), and arrows that started or ended at \(Decartes\) get pre- or post-composed with
the distinguished arrows \(dc\) and \(cd\).
Let's call this the \(\operatorname{fixup}\) map from \(X\) to \(X \setminus Decartes\).

Is \(\operatorname{fixup}\) a functor? Yes.

\begin{proof}
Suppose we had \(a_1; a_2 = a_3\) in \(X\).

Consider \(\operatorname{fixup} a_1; \operatorname{fixup} a_2\).
There are a few cases according to whether \(a_1\) or \(a_2\) or both had source or target or both equal to \(Decartes\),
but they're all very similar.

Note we are using brace expansion\footnote{https://www.linuxjournal.com/content/bash-brace-expansion} again.

\begin{align*}
\operatorname{fixup} a_1; \operatorname{fixup} a_2 & = \{ cd, id \}; a_1; \{ cd;dc, id \}; a_2; \{ dc, id \} && \text{(by definition of \(\operatorname{fixup}\))} \\
& = \{ cd, id \}; a_1; a_2; \{ dc, id \} && \text{(since \(cd; dc = id\))} \\
& = \{ cd, id \}; a_3; \{ dc, id \} && \text{(since \(a_1; a_2 = a_3\))} \\                         
& = \operatorname{fixup} a3 && \text{(by definition of \(\operatorname{fixup}\), reversed)}
\end{align*}
%\begin{align*}
%  4xyzw 
%  &= 2\cdot2tu 
%  \\ &\le 2\cdot(t^2+u^2)                    \tag{a remark in parentheses}
%  \\ &= 2\cdot((xy)^2+(zw)^2)
%  \\ &= 2\cdot(x^2y^2+z^2w^2)                \tag*{a remark without parentheses}
%  \\ &= 2x^2y^2+2z^2w^2
%  \\ &\le ((x^2)^2+(y^2)^2)+((z^2)^2)+(w^2)^2)
%  \\ &= x^4+y^4+z^4+w^4                      \qedhere
%\end{align*}

We have shown that \(a_1; a_2 = a_3\) implies \(\operatorname{fixup} a_1; \operatorname{fixup} a_2 = \operatorname{fixup} a_3\). So \(\operatorname{fixup}\) is a functor.
\end{proof}

The inclusion from \(X \setminus Decartes\) back to \(X\) is also a functor, which we might call \(\iota\).

\begin{proof}
If \(a_1; a_2 = a_3\) in \(X \setminus Decartes\), then \(\iota a_1; \iota a_2 = \iota a_3\) in \(X\),
because we didn't add or change any arrow compositions,
we only dropped some to create \(X \setminus Decartes\).
\end{proof}

Is there a natural transformation from \(\operatorname{fixup}; \iota\) to \(\operatorname{id}\)?

What does it mean to say there is a natural transformation from \(\operatorname{fixup}; \iota\) to \(\operatorname{id}\)?
There exists an \(n\) from objects of \(X\) to arrows of \(X\)
such that if you take a generic arrow \(a\) of \(X\)
then this square commutes:
\[
(\operatorname{fixup}; \iota) a; n (\operatorname{target} a) = n (\operatorname{source} a); (\operatorname{fixup}; \iota) a
\]

Fixup followed by \(\iota\) slides arrows that might start or finish at \(Decartes\) to arrows that instead start or finish at \(Cross\).
So in order for the types to match up, \(n\) should take objects to endo-arrows, except \(Decartes\), which should go to an endo-arrow on \(Cross\).
Let's try defining \(n\) to take everything to its own identity arrow,
except \(Decartes\), which will go to \(Cross\)'s identity arrow. 

This \(n\) turns out to be a natural transformation from \(\operatorname{fixup}; \iota\) to \(id\).

\begin{proof}
\begin{align*}
(\operatorname{fixup}; \iota) a; n (\operatorname{target} a) & = \iota (\operatorname{fixup} a); n (\operatorname{target} a) \\
& = \iota (\{ 1, cd \}; a; \{ dc, 1 \}); n (\operatorname{target} a) \\      
& = \{ 1, cd \}; a; \{ dc, 1 \}; n (\operatorname{target} a) \\
& = \{ 1, cd \}; a; \{ dc, 1 \}; \{ \operatorname{id} Cross, 1 \} \\
& = \{ 1, cd \}; a; \{ dc, 1 \} \\
& = \{ 1, \operatorname{id} Cross \}; \{ 1, cd \}; a; \{ dc, 1 \} \\
& = \{ 1, \operatorname{id} Cross \}; \iota (\{ 1, cd \}; a; \{ dc, 1 \}) \\
& = \{ 1, \operatorname{id} Cross \}; \iota (\operatorname{fixup} a) \\
& = \{ 1, \operatorname{id} Cross \}; (\operatorname{fixup}; \iota) a \\   
& = n (\operatorname{source} a); (\operatorname{fixup}; \iota) a   \qedhere
\end{align*}
\end{proof}

Is there a natural transformation from \(\iota; \operatorname{fixup}\) to \(\operatorname{id}\)? Yes.

\begin{proof}
\(\iota; \operatorname{fixup}\) is an inclusion followed by a function which is the identity on the image of the inclusion,
so it's equal to the identity already.
\end{proof}

Moreover, these natural transformations are actually natural isomorphisms; they always take objects to identity arrows, and identity arrows are always isomorphisms. So \(X\) is actually equivalent to \(X \setminus Decartes\).

We haven't actually used anything specific about \(Decartes\) in these two conclusions, so we also have the analogous maps from \(X\) to \(X \setminus Cross\) and from \(X \setminus Cross\) back to \(X\).

So \(X \setminus Decartes\) is equivalent as a category to \(X \setminus Cross\); it's not possible to `see' the substitution from a distance via any arrows composing differently.


\end{document}
