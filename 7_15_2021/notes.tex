\DeclareSymbolFont{AMSb}{U}{msb}{m}{n}
\documentclass[11pt,noamsfonts]{amsart}
\usepackage[left=1.5in, right=1.5in, top=1.2in, bottom=1.2in]{geometry}
\usepackage{mathtools}
\usepackage{braket}
\usepackage{enumitem}
\usepackage[charter,expert]{mathdesign}
\usepackage[scaled=.96,osf]{XCharter}% matches the size used in math
\usepackage[tracking]{microtype}
\usepackage{tikz-cd}
\usepackage{stmaryrd}
\usepackage{comment}
\usepackage[scr=esstix]{mathalfa}

\usepackage{caption}
\usepackage{subcaption}

\usepackage{hyperref}
\linespread{1.04}

\usepackage{enumitem}
\setlist[1]{labelindent=\parindent}
\setlist[enumerate]{labelsep=0.5em}
\setlist[enumerate,1]{label={\upshape (\roman*)}, ref={\upshape (\roman*)}}
\setlist[itemize]{label={--}}

\DeclareMathOperator{\id}{id}
\DeclareMathOperator{\pr}{pr}

\makeatletter
\let\c@equation\c@section
\let\theequation\thesection
\makeatother

%found on https://tex.stackexchange.com/a/565122 with improvements from TikZ manual:
\tikzset{>={Straight Barb[length=2pt,width=4pt]}, commutative diagrams/arrow style=tikz}

\usepackage[utf8]{inputenc}

\newcommand{\todo}[1]{\footnote{{\sc\color{red}Todo.} #1}}

% Point Formats
\newcommand{\pointheader}{\vspace{2mm}\noindent\refstepcounter{section}\textbf{\thesection.}}
\newcommand{\point}{\pointheader~}
\newcommand{\tpoint}[1]{\pointheader~{\bf #1. ---}}
\newcommand{\epoint}[1]{\pointheader~{\em #1.}}
\newcommand{\bpoint}[1]{\pointheader~{\bf #1.}}

% QED Symbol
\newcommand{\psqedsymb}{\(\blacksquare\)}
\renewcommand{\qed}{~\hfill{\psqedsymb}}


\makeatletter
\newcommand*{\coloneqq}{\mathrel{\rlap{%
           \raisebox{0.3ex}{$\m@th\cdot$}}%
           \raisebox{-0.3ex}{$\m@th\cdot$}}%
           =}
\newcommand{\eqqcolon}{=%
           \mathrel{\rlap{%
           \raisebox{0.3ex}{$\m@th\cdot$}}%
           \raisebox{-0.3ex}{$\m@th\cdot$}}}
\makeatother

\DeclareMathOperator{\Hom}{Hom}

\title{Notes for July 15, 2021}
\begin{document}
\maketitle

\bpoint{Fibre Products}
Let \(\mathcal{C}\) be a category and let \(f \colon X \to Z\) and \(g \colon Y \to Z\)
be morphisms in \(\mathcal{C}\). A \emph{fibre product of \(X\) and \(Y\) over \(Z\)} (with respect
to \(f\) and \(g\)) is an object \(P\) together with morphisms \(p \colon P \to X\)
and \(q \colon P \to Y\) that fit into a commutative diagram
\[
\begin{tikzcd}
P \ar[r,"p"] \ar[d,"q"'] & X \ar[d,"f"] \\
Y \ar[r,"g"'] & Z
\end{tikzcd}
\quad\text{that is,}\; f \circ p = g \circ q,
\]
such that if \(P'\) is any object with morphisms \(p' \colon P' \to X\) and \(q' \colon P' \to Y\)
fitting into a commutative square as above---\(f \circ p' = g \circ q'\) as morphisms
\(P' \to Z\)---then there exists a unique morphism \(h \colon P' \to P\) such that
the triangles below commute
\[
\begin{tikzcd}
  & P' \ar[d,"h"] \ar[dl,"q'"'] \ar[dr,"p'"] \\
Y & \ar[l,"q"] P \ar[r,"p"'] & X
\end{tikzcd}
\quad\text{that is,}\; p' = p \circ h\;\text{and}\; q' = q \circ h.
\]
A fibre product \(P\) is sometimes denoted by \(X \times_Z Y\), or to make
the dependence on \(f\) and \(g\) explicit, \(X \times_{f,Z,g} Y\).

\point
I think of this as ``taking the product of \(X\) and \(Y\) with equations
imposed by \(f\) and \(g\).'' For example, when \(\mathcal{C} = \mathrm{Set}\),
then the fibre product above can be explicitly identified as
\[ X \times_{f,Z,g} Y = \set{(x,y) \in X \times Y | f(x) = g(y)}, \]
that is, this consists of all pairs in the usual Cartesian\footnote{This is the
same reason why one says that the fibre product diagram above is \emph{Cartesian}!}
product such that they become equal after mapping to \(Z\). In particular,
taking \(Z = \{*\}\) the singleton---the terminal object in the category of sets!---we
recover the usual Cartesian product:
\[ X \times_{\{*\}} Y = X \times Y \]
since everything in \(X\) and \(Y\) must map to the single element \(*\).

\bpoint{Exercise.}\label{fibre-product-exercise}
Compare the diagram defining the Cartesian product and the fibre product
diagrams. Prove that if a category has a terminal object, then the product is a
fibre product over the terminal object. Prove that a morphism \(Z \to W\) in
\(\mathcal{C}\) induces a morphism
\[ X \times_{Z} Y \to X \times_{W} Y. \]
In particular, if a category has a terminal object, any fibre product maps to
the Cartesian product.

\bpoint{Pushouts}
As is true about every concept in category theory, there is a dual notion whereby
all arrows are reversed. Let \(i \colon Z \to X\) and \(j \colon Z \to Y\) be morphisms
in a category \(\mathcal{C}\). Then the \emph{pushout of \(X\) and \(Y\) over \(Z\)}
(with respect to \(i\) and \(j\)) is an object \(Q\) together with morphisms
\(u \colon X \to Q\) and \(v \colon Y \to Q\) that fit into a commutative diagram
\[
\begin{tikzcd}
Z \ar[r,"i"] \ar[d,"j"'] & X \ar[d,"u"] \\
Y \ar[r,"v"'] & Q
\end{tikzcd}
\quad\text{that is,}\; u \circ i = v \circ j,
\]
such that if \(Q'\) is any object with morphisms \(u' \colon X \to Q'\) and \(v' \colon Y \to Q'\)
fitting into a commutative square as above---\(u' \circ i = v' \circ j\) as morphisms \(Z \to Q'\)---then
there exists a unique morphism \(k \colon Q \to Q'\) such that the triangles below commute:
\[
\begin{tikzcd}
Y \ar[r,"v"] \ar[dr,"v'"'] & Q \ar[d,"k"] & \ar[l,"u"'] \ar[dl,"u'"] X \\
& Q'
\end{tikzcd}
\quad\text{that is,}\; u' = k \circ u \;\text{and}\; v' = k \circ v.
\]
The pushout \(Q\) is sometimes written \(X \sqcup_{Z} Y\) or \(X \sqcup_{i,Z,j} Y\)
to make the dependence on \(i\) and \(j\) explicit.

\point
I think of this as ``glueing \(X\) and \(Y\) along \(Z\).'' For example, when
\(\mathcal{C} = \mathrm{Set}\), then the pushout can be explicitly computed as
\[
X \sqcup_{i,Z,j} Y = (X \sqcup Y)/(i(z) \sim j(z) \;\text{for all \(z \in Z\)}).
\]
That is, you first take the disjoint union \(X \sqcup Y\) of \(X\) and \(Y\),
and then you form the quotient set where you identify all elements of \(X\)
and \(Y\) that come from the same element of \(Z\).

For an explicit example, let \(X = \set{1,2,3}\) and \(Y = \set{a,b,c,d}\).
Let \(Z = \set{*}\) be the singleton, and let \(i \colon Z \to X\) be \(* \mapsto 1\),
and let \(j \colon Z \to Y\) be \(* \mapsto a\). Then
\[
X \sqcup_{i,Z,j} Y = \set{1,2,3,a,b,c,d}/(1 \sim a) = \set{2,3,b,c,d,\spadesuit}
\]
where \(\spadesuit\) is the equivalence class of \(1\) and \(a\); alternatively,
you might think of \(\spadesuit = \{1,a\}\).

\bpoint{Exercise}
Formulate and do the co-Exercise to \ref{fibre-product-exercise}.

\bpoint{Vector Spaces}
Let \(k\) be a field, which you may think of \(k = \mathbf{R}\) to be concrete.
Let \(\mathrm{Vect}\) be the category of finite dimensional vector spaces over
\(k\). Let's think about some properties of this category. Here are a list of
claims that you might enjoy thinking through: Let \(U\), \(V\), and \(W\)
be finite dimensional vector spaces over \(k\).
\begin{enumerate}
\item The initial and terminal object of \(\mathrm{Vect}\) is the vector space
\(\{0\}\).
\item The product and coproduct of \(U\) and \(V\) are given by
\[
U \times V = U \sqcup V \eqqcolon U \oplus V \coloneqq \set{(u,v) | u \in U, v \in V}
\]
where the structure of a vector space on \(U \oplus V\) is given by
\[
(u,v) + (u',v') = (u + u', v + v')
\quad\text{and}\quad
\lambda(u,v) = (\lambda u, \lambda v).
\]
\item Let \(f \colon U \to W\) and \(g \colon V \to W\) be linear maps. Then
\[ U \times_{f,W,g} V = \set{(u,v) \in U \oplus V | f(u) = g(v)}. \]
\item Let \(i \colon U \to V\) and \(j \colon U \to W\) be linear maps. Then
\[ V \sqcup_{i,U,j} W = (V \oplus W)/\set{(i(u), j(u)) | u \in U}. \]
In other words, \(V \sqcup_{i,U,j} W\) is the quotient vector space on \(V \oplus W\)
divided by the image of the map \((i,j) \colon U \to V \oplus W\).
\end{enumerate}

\bpoint{Exercise}
Let \(f \colon U \to W\) be a linear map. Express the \emph{kernel} of \(f\),
\[ \ker(f) \coloneqq \set{u \in U | f(u) = 0} \]
as a fibre product. Likewise, let \(i \colon U \to V\) be a linear map. Express
the \emph{cokernel} of \(i\)
\[ \operatorname{coker}(i) \coloneqq V/\set{i(u) | u \in U} \]
as a pushout.

\bpoint{Tensor Products}
Apparently, the product and coproduct of two vector spaces match, and are
given by the direct sum \(V \oplus W\). Using the relations above, you can
prove that if
\begin{enumerate}
\item \(\set{v_1,\ldots,v_n}\) is a basis for \(V\), and
\item \(\set{w_1,\ldots,w_m}\) is a basis for \(W\),
\end{enumerate}
then \(\set{v_1,\ldots,v_n,w_1,\ldots,w_m}\) is a basis for \(V \oplus W\).
In particular, this means that
\[ \dim(V \oplus W) = \dim(V) + \dim(W) \]
there is an additive relation on the dimension of direct sums. In this sense,
\(V \oplus W\) is thought of as the sum of the vector spaces \(V\) and \(W\).

Then \emph{tensor product} \(V \otimes W\) of \(V\) and \(W\) is another vector
space which is better thought of as the product of \(V\) and \(W\); in
particular, there is a multiplicative relation on dimensions:
\[ \dim(V \otimes W) = \dim(V) \dim(W). \]
One way to define the tensor product is as the vector space\footnote{Meaning that
you are allowed to take \(k\)-linear combinations of the symbols!} on the set
of symbols
\[ \set{v \otimes w | v \in V, w \in W} \]
subject to the following relations: for all \(v, v' \in V\), \(w, w' \in W\),
and \(\lambda \in k\),
\begin{enumerate}
\item \(v \otimes w + v' \otimes w = (v + v') \otimes w\),
\item \(v \otimes w + v \otimes w' = v \otimes (w + w')\),
\item \((\lambda v) \otimes w = \lambda (v \otimes w) = v \otimes (\lambda w)\).
\end{enumerate}
From these, one can prove that a basis for \(V \otimes W\) is given by
\[ \set{v_i \otimes w_j | i = 1,\ldots,n \;\text{and}\; j = 1,\ldots,m } \]
where \(\set{v_1,\ldots,v_n}\) and \(\set{w_1,\ldots,w_m}\) are bases for \(V\)
and \(W\), as above.\footnote{A word of caution: not all elements of
\(V \otimes W\) are of the form \(v \otimes w\). It may well be the case that
an element \(v \otimes w + v' \otimes w'\)  cannot be combined together as
\(v'' \otimes w''\).}

\bpoint{Symmetric Monoidal Categories}
The tensor product operation is \emph{not} internal to the category
\(\mathrm{Vect}\): it is extra \emph{structure} that we can endow
\(\mathrm{Vect}\). The pair \((\mathrm{Vect}, \otimes)\) is the prototype of
a \emph{symmetric monoidal category}. In general, this is a category \(\mathcal{C}\)
together with a binary operation
\[ \otimes \colon \mathcal{C} \times \mathcal{C} \to \mathcal{C} \]
that is ``commutative'' and ``has a unit''. One has to be a bit careful to state
this, since in the setting of categories, you do not necessarily want an \emph{equality}
between \(A \otimes B\) and \(B \otimes A\), but rather an isomorphism. Then there
are conditions that you have to impose on this isomorphism to make it sensible.
See the Wikipedia page \href{https://en.wikipedia.org/wiki/Symmetric_monoidal_category}{here}
for some details, but the short of it is:
a symmetric monoidal category is a category \(\mathcal{C}\) with a operation
\(\otimes\) which allows you to multiply objects together.

\bpoint{Example}
The \(\otimes\) does not need to be really extra structure. For instance,
\(\mathcal{C} = \mathrm{Set}\) with \(\otimes = \times\), the usual Cartesian
product, makes the category of sets into a symmetric monoidal category.

\bpoint{Monoids}
A \emph{monoid} is a set \(M\) with a binary operation \(\mu \colon M \times M \to M\)
such that:
\begin{enumerate}
\item There is a unit: there exists \(e \in M\) such that \(\mu(e,m) = \mu(m, e) = m\)
for all \(m \in M\).
\item It is associative: \(\mu(m, \mu(n, p) = \mu(\mu((m, n), p\) for all
\(m,n,p \in M\).
\end{enumerate}

Now let \((\mathcal{C},\otimes)\) be a (symmetric)\footnote{Symmetry, that is, that
\(\otimes\) is ``commutative'', is not strictly necessary here.} monoidal
category with a terminal object \(\mathbf{1}\). A \emph{monoid object} in
\(\mathcal{C}\) is an object \(\mathbf{M}\) in \(\mathcal{C}\) together with
morphisms
\[ e \colon \mathbf{1} \to \mathbf{M} \quad\text{and}\quad  \mu \colon \mathbf{M} \otimes \mathbf{M} \to \mathbf{M} \]
such that the following two diagrams commute:
\[
\begin{tikzcd}
\mathbf{1} \otimes \mathbf{M} \ar[r,"e \otimes \id_\mathbf{M}"] \ar[dr,"\mathrm{pr}_2"']  & \mathbf{M} \otimes \mathbf{M} \dar["\mu"] & \ar[l,"\id_\mathbf{M} \otimes e"'] \ar[dl,"\pr_1"]  \mathbf{M} \otimes \mathbf{1} \\
& \mathbf{M}
\end{tikzcd}
\quad\text{and}\quad
\begin{tikzcd}
\mathbf{M} \otimes \mathbf{M} \otimes \mathbf{M} \ar[r,"\mu \otimes \id_\mathbf{M}"] \ar[d,"\id_\mathbf{M} \otimes \mu"'] & \mathbf{M} \otimes \mathbf{M} \ar[d,"\mu"] \\
\mathbf{M} \otimes \mathbf{M} \ar[r,"\mu"'] & \mathbf{M}.
\end{tikzcd}
\]
The left diagram is the fact that \(e \colon \mathbf{1} \to M\) behaves like a
unit; and the right diagram is associativity of the multiplication operation
\(\mu\).

\bpoint{Exercise}\label{monoid-exercise}
Let \(\mathbf{M}\) be a monoid object in \(\mathcal{C}\). Let \(X\) be any other object
in \(\mathcal{C}\) and let \(\mathbf{M}(X) \coloneqq \mathcal{C}(X,\mathbf{M})\) be the
set of arrows from \(X\) to \(\mathbf{M}\) in \(\mathcal{C}\). Verify that \(e\) and \(\mu\)
induce maps \(e(X) \colon \mathbf{1}(X) \to \mathbf{M}(X)\) and \(\mu(X) \colon \mathbf{M}(X) \times \mathbf{M}(X) \to \mathbf{M}(X)\)
and show that \((\mathbf{M}(X), e(X), \mu(X))\) is a monoid in the usual sense.\footnote{Hint: \(\mathbf{1}\) is
a terminal object, so there is a only one map from anything to \(\mathbf{1}\).}

\bpoint{Modules and Algebras}
Let \((M,e,\mu)\) be a (plain old) monoid. A \emph{module} over \(M\) is an
set \(P\) together with an action map \(\alpha \colon M \times P \to P\) such that:
For every \(m,n \in M\) and \(p \in P\),
\begin{enumerate}
\item The action is compatible with multiplcation: \(\alpha(\mu(m,n),p) = \alpha(m,\alpha(n,p))\).
\item The identity acts trivially: \(\alpha(e,p) = p\).
\end{enumerate}
An \emph{algebra} over \(M\) is a module \(A\) over \(M\)
together with a binary operation \(\nu \colon A \times A \to A\) such that
for all \(a,b,c \in A\),
it commutes with the action of \(M\):
\[ \nu(\alpha(m,a), \alpha(n,b)) = \alpha(\mu(m,n),\nu(a,b)). \]
We may also ask that \(\nu\) is associative and/or has an identity element, or
place other conditions on \(\nu\).

Now let \((\mathcal{C},\otimes)\) be a symmetric monoidal category with
terminal object \(\mathbf{1}\) and finite products. Let \((\mathbf{M},e,\mu)\)
be a monoid object in \(\mathcal{C}\). A \emph{module} over \(\mathbf{M}\)
is an object \(\mathbf{P}\) of \(\mathcal{C}\) together with a morphism
\(\alpha \colon \mathbf{M} \otimes \mathbf{P} \to \mathbf{P}\) such that the
following diagrams commute:
\[
\begin{tikzcd}
\mathbf{M} \otimes \mathbf{M} \otimes \mathbf{P} \ar[r,"\mu \otimes \id_{\mathbf{P}}"] \ar[d,"\id_{\mathbf{M}} \otimes \alpha"'] &
\mathbf{M} \otimes \mathbf{P} \dar["\alpha"] \\
\mathbf{M} \otimes \mathbf{P} \rar["\alpha"'] & \mathbf{P}
\end{tikzcd}
\quad\text{and}\quad
\begin{tikzcd}
\mathbf{1} \otimes \mathbf{P} \ar[rr,"e \otimes \id_{\mathbf{P}}"] \ar[dr,"\pr_2"'] &&  \mathbf{M} \otimes \mathbf{P} \ar[dl,"\mu"] \\
& \mathbf{P}.
\end{tikzcd}
\]
The left diagram expresses the compatibility between the action map and the multiplication
of the monoid object, and the right expresses the fact that the identity does nothing.

\bpoint{Exercise}
Formulate and repeat Exercise \ref{monoid-exercise} for modules over monoid objects.
Finish the definition of an algebra over a monoid object, and also verify the
corresponding property on morphisms.

\end{document}
