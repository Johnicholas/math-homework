\DeclareSymbolFont{AMSb}{U}{msb}{m}{n}
\documentclass[11pt,noamsfonts]{amsart}
\usepackage[left=1.5in, right=1.5in, top=1.2in, bottom=1.2in]{geometry}
\usepackage{mathtools}
\usepackage{braket}
\usepackage{enumitem}
\usepackage[charter,expert]{mathdesign}
\usepackage[scaled=.96,osf]{XCharter}% matches the size used in math
\usepackage[tracking]{microtype}
\usepackage{tikz-cd}
\usepackage{stmaryrd}
\usepackage{comment}
\usepackage[scr=esstix]{mathalfa}

\usepackage{caption}
\usepackage{subcaption}

\usepackage{hyperref}
\linespread{1.04}

\usepackage{enumitem}
\setlist[1]{labelindent=\parindent}
\setlist[enumerate]{labelsep=0.5em}
\setlist[enumerate,1]{label={\upshape (\roman*)}, ref={\upshape (\roman*)}}
\setlist[itemize]{label={--}}

\DeclareMathOperator{\Sym}{Sym}

\makeatletter
\let\c@equation\c@section
\let\theequation\thesection
\makeatother

%found on https://tex.stackexchange.com/a/565122 with improvements from TikZ manual:
\tikzset{>={Straight Barb[length=2pt,width=4pt]}, commutative diagrams/arrow style=tikz}

\usepackage[utf8]{inputenc}

\newcommand{\todo}[1]{\footnote{{\sc\color{red}Todo.} #1}}

% Point Formats
\newcommand{\pointheader}{\vspace{2mm}\noindent\refstepcounter{section}\textbf{\thesection.}}
\newcommand{\point}{\pointheader~}
\newcommand{\tpoint}[1]{\pointheader~{\bf #1. ---}}
\newcommand{\epoint}[1]{\pointheader~{\em #1.}}
\newcommand{\bpoint}[1]{\pointheader~{\bf #1.}}

% QED Symbol
\newcommand{\psqedsymb}{\(\blacksquare\)}
\renewcommand{\qed}{~\hfill{\psqedsymb}}


\makeatletter
\newcommand*{\coloneqq}{\mathrel{\rlap{%
           \raisebox{0.3ex}{$\m@th\cdot$}}%
           \raisebox{-0.3ex}{$\m@th\cdot$}}%
           =}
\newcommand{\eqqcolon}{=%
           \mathrel{\rlap{%
           \raisebox{0.3ex}{$\m@th\cdot$}}%
           \raisebox{-0.3ex}{$\m@th\cdot$}}}
\makeatother

\DeclareMathOperator{\Hom}{Hom}

\title{Notes for July 8, 2021}
\begin{document}
\maketitle

\bpoint{Binomial Coefficients}
Let \(n\) and \(k\) be positive integers. Then the binomial coefficient
\[
\binom{n}{k} \coloneqq \frac{n!}{k!(n-k)!}
\]
has many combinatorial interpretations including:
\begin{enumerate}
\item it is the number of binary strings of length \(n\) with exactly
\(k\) appearances of \(1\);
\item it is the number of Dyck paths of length \(n\) with exactly \(k\)
upward steps;
\item it is the number of size \(k\) subsets of a size \(n\) set.
\end{enumerate}
One categorification of the binomial coefficient in the category of sets then
might be:
\begin{align*}
\binom{-}{-} \colon \mathrm{Set} \times \mathrm{Set} \to \mathrm{Set}
(N,K) & \mapsto \set{S \in \Hom(N,[2]) | S\;\text{is isomorphic to}\;K}.
\end{align*}
In words, this is the functor that takes a pair of finite sets \((N,K)\),
produces the set \(\Hom(N,[2])\) of all subsets of \(N\)---binary strings
correspond to subsets of \(N\)!---and then picks out the subsets with the same
size as \(K\).\footnote{Note, however, that in order to make this into an
honest functor, one might have to be careful about what morphisms to take in
the categories of sets in question.} This is a categorification of the binomial
coefficient in the sense that the following diagram commutes:
\[
\begin{tikzcd}
\mathrm{Set} \times \mathrm{Set} \ar[r,"\binom{-}{-}"] \ar[d,"\#"'] & \mathrm{Set} \ar[d,"\#"] \\
\mathbf{Z}_+ \times \mathbf{Z}_+ \ar[r,"\binom{-}{-}"] & \mathbf{Z}_+
\end{tikzcd}
\]
which means, explicitly, given finite sets \(N\) and \(K\) with sizes \(n \coloneqq \#N\)
and \(k \coloneqq \#K\), there is the relation
\[ \#\binom{N}{K} = \binom{\#N}{\#K} = \binom{n}{k}. \]

This might be carried out in other categories, too! The only thing that was
special to the category of sets was the construction of the set of all subsets
of \(N\). But this amounted to saying that the object \([2]\) had the property
that \(\Hom(N,[2])\) classified subobjects of \(N\): in other words, \([2]\)
is a \emph{subobject classifier} in the category of sets. So we might
perform the same construction in a general category \(\mathsf{C}\) by replacing
\([2]\) in the above formula with a subobject classifier \(\Omega\) in \(\mathsf{C}\).

\bpoint{Representations of Finite Groups}
Let \(G\) be a group. A \emph{representation of \(G\)} on a finite-dimensional
complex vector space \(V\) is a linear action of \(G\) on \(V\): that is, this
is a binary operator \(\cdot \colon G \times V \to V\)
such that for every \(g, h \in G\), \(v,w \in V\), and \(\lambda \in \mathbf{C}\),
\begin{enumerate}
\item \(g \cdot (v + \lambda w) = (g \cdot v) + \lambda (g \cdot w)\),
\item \(h \cdot (g \cdot v) = (hg) \cdot v\), and
\item \(e \cdot v = v\),
\end{enumerate}
where \(e \in G\) is the identity element. Equivalently, let
\[ \mathrm{GL}(V) \coloneqq \set{T \colon V \to V | T\;\text{an invertible linear transformation}}, \]
then the data of a representation of \(G\) on \(V\) amounts to a
homomorphism \(\rho \colon G \to \mathrm{GL}(V)\).\footnote{Note that when \(V =
\mathbf{C}^{\oplus n}\) is the standard \(n\)-dimensional vector space, then
\(\mathrm{GL}(V) = \mathrm{GL}_n(\mathbf{C})\) is none other than the group of
invertible square complex matrices of size \(n\): this is the correspondence
between linear transformations and matrices, upon picking a basis.} A subspace
\(W \subseteq V\) is a \emph{subrepresentation} if
\[ g \cdot w \in W \quad\text{for all}\;g \in G\;\text{and all}\; w \in W. \]
A representation \(V\) is called \emph{irreducible} if whenever \(W \subseteq V\)
is a subrepresentation, then either \(W = \{0\}\) or \(W = V\).

The basic problem of representation theory is to classify irreducible
representations of finite groups \(G\), the point being that irreducible
representations are the building blocks of all representations. Over the complex
numbers, all representations are built out of irreducible ones in the simplest
possible way:

\tpoint{Maschke's Theorem}
\emph{Let \(G\) be a finite group. Then for every complex representation \(V\),
there is a decomposition
\[ V \cong \bigoplus\nolimits_{i \in I} V_i \]
where \(I\) is a finite index set, and the \(V_i\) are irreducible
representations of \(G\).}

\point
Moreover, the essential data of an irreducible representation turns out to be
something quite intrinsic to \(G\). Namely, let \(V\) be a representation of
\(G\), with associated homomorphism \(\rho \colon G \to \mathrm{GL}(V)\).
The \emph{character} of this representation is the homomorphism
\[ \chi \coloneqq \mathrm{trace} \circ \rho \colon G \to \mathrm{GL}(V) \to \mathbf{C} \]
obtained by taking an element \(g \in G\), considering the linear
transformation \(\rho(g) \in \mathrm{GL}(V)\), choosing any basis of \(V\) to
make \(\rho(g)\) into a complex matrix, and then taking the trace of the
resulting matrix. The properties of the trace ensure that this is independent
of the choice of basis chosen, and hence this is really an invariant of the
representation \(\rho\) itself. Then the classification theorem for irreducible
representations is

\tpoint{Character Theory of Finite Groups}
\emph{Let \(G\) be a finite group. Let \(V\) and \(W\) be irreducible
representations of \(G\) with characters \(\chi\) and \(\psi\), respectively.
Then \(V\) and \(W\) are isomorphic representations if and only if \(\chi\) and
\(\psi\) are the same function on \(G\).}

\point
Thus the problem of classifying irreducible representations of \(G\) is greatly
aided by the characters. Here are a few more facts:
\begin{enumerate}
\item characters are \emph{class functions}, that is, if \(g, h \in G\)
are in the same \href{https://en.wikipedia.org/wiki/Conjugacy_class}{conjugacy
class}\footnote{Read: if up to a change of labels/coordinates, they are the
same.}, then \(\chi(g) = \chi(h)\);
\item the set of all complex-valued class functions out of \(G\) is a
vector space with dimension equal to the number of conjugacy classes of \(G\):
take the indicator function for each conjugacy class;
\item there is an inner product on the space of class functions;
\item the irreducible characters form an orthogonal basis for the space of
class functions;
\item each irreducible representations of \(G\) is a direct summand, appearing
exactly once, of the \emph{regular representation} of \(G\):
\[ \mathbf{C}G \coloneqq \Set{\sum\nolimits_{g \in G} a_g g | a_g \in \mathbf{G}} \]
this is the vector space spanned by symbols indexed by elements of \(G\) and
\(G\) acts on this by multiplication on the symbols.
\end{enumerate}

\bpoint{Young Diagrams, Tableaux}
A \href{https://en.wikipedia.org/wiki/Young_tableau}{\emph{Young diagram}} is a
set of boxes that is justified in some fashion. There are many conventions, but
let's use the so-called English convention in which the boxes are left and upper
aligned. The number of boxes in each row of a Young diagram \(\mathsf{Y}\)
form a partition  \(\lambda = (\lambda_1 \geq \lambda_2 \geq \lambda_3 \geq \ldots)\),
that is, a non-increasing sequence of nonnegative integers. The relevance of
Young diagrams appears in their connection with the representation theory of
the symmetric group: the set of irreducible representations of the symmetric
group \(S_n\) on \(n\) elements are naturally indexed by a Young diagram with \(n\)
boxes. In fact, given a Young diagram \(\mathsf{Y}\) with \(n\) boxes, the
corresponding \href{https://en.wikipedia.org/wiki/Specht_module}{irreducible
representation} can be made by making a vector space whose basis consists of
certain fillings of \(\mathsf{Y}\).

Another related thing involves \href{https://en.wikipedia.org/wiki/Symmetric_function}{symmetric functions}.
There are certain symmetric polynomials called the
\href{https://en.wikipedia.org/wiki/Schur_polynomial}{Schur polynomials}, and
they can be specified by taking and counting certain Young diagrams. These are
related to the irreducible polynomial representations of the general linear
group \(\mathrm{GL}_n(\mathbf{C})\), which in turn is related to the symmetric
group situation via what is known as \href{https://en.wikipedia.org/wiki/Schur%E2%80%93Weyl_duality}{Schur--Weyl duality}.
Everything is connected!

\bpoint{Universal Invariants}
Finally we discussed the notion of finding some sort of universal way of making
an algebraic invariant for some type of object, satisfying some set of
relations. The one that I know best is the notion of an \emph{Euler characteristic}.
The basic example would be: I am looking for an invariant \(\chi\) on the set
\(\mathrm{Vect}\) of finite-dimensional vector spaces such that
if I can write \(V = V_1 \oplus V_2\), then
\[ \chi(V) = \chi(V_1) + \chi(V_2). \]
Well, what sort of structure do I need to make sense of this? I just need to
be able to add my invariants, so I am probably looking for a function
\[ \chi \colon \mathrm{Vect} \to G \]
for some abelian group \(G\). I want this to be \emph{universal} in the sense
that if I have any other map \(\psi \colon \mathrm{Vect} \to H\) to some
other abelian group \(H\), then I would like there to be a unique homomorphism
\(\bar\psi \colon G \to H\) such that the following diagram commutes:
\[
\begin{tikzcd}
\mathrm{Vect}\rar["\chi"] \ar[dr,"\psi"']  & G \ar[d,"\bar\psi"] \\ & H
\end{tikzcd}
\quad\text{that is,}\; \psi = \bar\psi \circ \chi.
\]
To prove that this exists, I need to construct such an object. And the right
thing to do is the least lossy thing possible: take \(G\) to be the free
abelian group generated on isomorphism classes of objects on \(\mathrm{Vect}\),
subject to the additive relations. Explicitly, the isomorphism classes of objects
in \(\mathrm{Vect}\) are represented by the standard vector spaces \(\mathbf{C}^n\)
of dimension \(n\), for each integer \(n \geq 0\). Then I would like to set \(G\) to be
the abelian group generated by symbols \([\mathbf{C}^n]\). But I need to
subject these to the direct sum relation. Note, however, that
\[
\mathbf{C}^n
= \mathbf{C}^{n-1} \oplus \mathbf{C}
= \cdots \mathbf{C} \oplus \mathbf{C} \oplus \cdots \oplus \mathbf{C}
\]
by peeling off coordinates. So this means that I need to impose the relations
\[ [\mathbf{C}^n] = [\mathbf{C}] + [\mathbf{C}] + \cdots + [\mathbf{C}] = n[\mathbf{C}]. \]
Hence \(G\) is the free abelian group generated on the single symbol \([\mathbf{C}]\), and hence
is the group of integers \(G \cong \mathbf{Z} \cdot [\mathbf{C}]\)
and the function \(\chi \colon \mathrm{Vect} \to G\) is none other than the
dimension function!

\point
This construction can be repeated when the category of vector spaces is replaced
by any other category \(\mathcal{A}\) in which some notion of ``an object is
the sum of two others'' is represent. This is axiomitized in the notion of a
\href{https://en.wikipedia.org/wiki/Abelian_category}{abelian category}
and the resulting construction is known as the
\href{https://en.wikipedia.org/wiki/Grothendieck_group}{Grothendieck group} of
the associated category.



\end{document}
