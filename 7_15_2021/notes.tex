\DeclareSymbolFont{AMSb}{U}{msb}{m}{n}
\documentclass[11pt,noamsfonts]{amsart}
\usepackage[left=1.5in, right=1.5in, top=1.2in, bottom=1.2in]{geometry}
\usepackage{mathtools}
\usepackage{braket}
\usepackage{enumitem}
\usepackage[charter,expert]{mathdesign}
\usepackage[scaled=.96,osf]{XCharter}% matches the size used in math
\usepackage[tracking]{microtype}
\usepackage{tikz-cd}
\usepackage{stmaryrd}
\usepackage{comment}
\usepackage[scr=esstix]{mathalfa}

\usepackage{caption}
\usepackage{subcaption}

\usepackage{hyperref}
\linespread{1.04}

\usepackage{enumitem}
\setlist[1]{labelindent=\parindent}
\setlist[enumerate]{labelsep=0.5em}
\setlist[enumerate,1]{label={\upshape (\roman*)}, ref={\upshape (\roman*)}}
\setlist[itemize]{label={--}}

\DeclareMathOperator{\id}{id}
\DeclareMathOperator{\pr}{pr}

\makeatletter
\let\c@equation\c@section
\let\theequation\thesection
\makeatother

%found on https://tex.stackexchange.com/a/565122 with improvements from TikZ manual:
\tikzset{>={Straight Barb[length=2pt,width=4pt]}, commutative diagrams/arrow style=tikz}

\usepackage[utf8]{inputenc}

\newcommand{\todo}[1]{\footnote{{\sc\color{red}Todo.} #1}}

% Point Formats
\newcommand{\pointheader}{\vspace{2mm}\noindent\refstepcounter{section}\textbf{\thesection.}}
\newcommand{\point}{\pointheader~}
\newcommand{\tpoint}[1]{\pointheader~{\bf #1. ---}}
\newcommand{\epoint}[1]{\pointheader~{\em #1.}}
\newcommand{\bpoint}[1]{\pointheader~{\bf #1.}}

% QED Symbol
\newcommand{\psqedsymb}{\(\blacksquare\)}
\renewcommand{\qed}{~\hfill{\psqedsymb}}


\makeatletter
\newcommand*{\coloneqq}{\mathrel{\rlap{%
           \raisebox{0.3ex}{$\m@th\cdot$}}%
           \raisebox{-0.3ex}{$\m@th\cdot$}}%
           =}
\newcommand{\eqqcolon}{=%
           \mathrel{\rlap{%
           \raisebox{0.3ex}{$\m@th\cdot$}}%
           \raisebox{-0.3ex}{$\m@th\cdot$}}}
\makeatother

\DeclareMathOperator{\Hom}{Hom}

\title{Notes for July 15, 2021}
\begin{document}
\maketitle

\bpoint{Fibre Products}
Let \(\mathcal{C}\) be a category and let \(f \colon X \to Z\) and \(g \colon Y \to Z\)
be morphisms in \(\mathcal{C}\). A \emph{fibre product of \(X\) and \(Y\) over \(Z\)} (with respect
to \(f\) and \(g\)) is an object \(P\) together with morphisms \(p \colon P \to X\)
and \(q \colon P \to Y\) that fit into a commutative diagram
\[
\begin{tikzcd}
P \ar[r,"p"] \ar[d,"q"'] & X \ar[d,"f"] \\
Y \ar[r,"g"'] & Z
\end{tikzcd}
\quad\text{that is,}\; f \circ p = g \circ q,
\]
such that if \(P'\) is any object with morphisms \(p' \colon P' \to X\) and \(q' \colon P' \to Y\)
fitting into a commutative square as above---\(f \circ p' = g \circ q'\) as morphisms
\(P' \to Z\)---then there exists a unique morphism \(h \colon P' \to P\) such that
the triangles below commute
\[
\begin{tikzcd}
  & P' \ar[d,"h"] \ar[dl,"q'"'] \ar[dr,"p'"] \\
Y & \ar[l,"q"] P \ar[r,"p"'] & X
\end{tikzcd}
\quad\text{that is,}\; p' = p \circ h\;\text{and}\; q' = q \circ h.
\]
A fibre product \(P\) is sometimes denoted by \(X \times_Z Y\), or to make
the dependence on \(f\) and \(g\) explicit, \(X \times_{f,Z,g} Y\).

\point
I think of this as ``taking the product of \(X\) and \(Y\) with equations
imposed by \(f\) and \(g\).'' For example, when \(\mathcal{C} = \mathrm{Set}\),
then the fibre product above can be explicitly identified as
\[ X \times_{f,Z,g} Y = \set{(x,y) \in X \times Y | f(x) = g(y)}, \]
that is, this consists of all pairs in the usual Cartesian\footnote{This is the
same reason why one says that the fibre product diagram above is \emph{Cartesian}!}
product such that they become equal after mapping to \(Z\). In particular,
taking \(Z = \{*\}\) the singleton---the terminal object in the category of sets!---we
recover the usual Cartesian product:
\[ X \times_{\{*\}} Y = X \times Y \]
since everything in \(X\) and \(Y\) must map to the single element \(*\).

\bpoint{Exercise.}\label{fibre-product-exercise}
Compare the diagram defining the Cartesian product and the fibre product
diagrams. Prove that if a category has a terminal object, then the product is a
fibre product over the terminal object. Prove that a morphism \(Z \to W\) in
\(\mathcal{C}\) induces a morphism
\[ X \times_{Z} Y \to X \times_{W} Y. \]
In particular, if a category has a terminal object, any fibre product maps to
the Cartesian product.

\bpoint{Pushouts}
As is true about every concept in category theory, there is a dual notion whereby
all arrows are reversed. Let \(i \colon Z \to X\) and \(j \colon Z \to Y\) be morphisms
in a category \(\mathcal{C}\). Then the \emph{pushout of \(X\) and \(Y\) over \(Z\)}
(with respect to \(i\) and \(j\)) is an object \(Q\) together with morphisms
\(u \colon X \to Q\) and \(v \colon Y \to Q\) that fit into a commutative diagram
\[
\begin{tikzcd}
Z \ar[r,"i"] \ar[d,"j"'] & X \ar[d,"u"] \\
Y \ar[r,"v"'] & Q
\end{tikzcd}
\quad\text{that is,}\; u \circ i = v \circ j,
\]
such that if \(Q'\) is any object with morphisms \(u' \colon X \to Q'\) and \(v' \colon Y \to Q'\)
fitting into a commutative square as above---\(u' \circ i = v' \circ j\) as morphisms \(Z \to Q'\)---then
there exists a unique morphism \(k \colon Q \to Q'\) such that the triangles below commute:
\[
\begin{tikzcd}
Y \ar[r,"v"] \ar[dr,"v'"'] & Q \ar[d,"k"] & \ar[l,"u"'] \ar[dl,"u'"] X \\
& Q'
\end{tikzcd}
\quad\text{that is,}\; u' = k \circ u \;\text{and}\; v' = k \circ v.
\]
The pushout \(Q\) is sometimes written \(X \sqcup_{Z} Y\) or \(X \sqcup_{i,Z,j} Y\)
to make the dependence on \(i\) and \(j\) explicit.

\point
I think of this as ``glueing \(X\) and \(Y\) along \(Z\).'' For example, when
\(\mathcal{C} = \mathrm{Set}\), then the pushout can be explicitly computed as
\[
X \sqcup_{i,Z,j} Y = (X \sqcup Y)/(i(z) \sim j(z) \;\text{for all \(z \in Z\)}).
\]
That is, you first take the disjoint union \(X \sqcup Y\) of \(X\) and \(Y\),
and then you form the quotient set where you identify all elements of \(X\)
and \(Y\) that come from the same element of \(Z\).

For an explicit example, let \(X = \set{1,2,3}\) and \(Y = \set{a,b,c,d}\).
Let \(Z = \set{*}\) be the singleton, and let \(i \colon Z \to X\) be \(* \mapsto 1\),
and let \(j \colon Z \to Y\) be \(* \mapsto a\). Then
\[
X \sqcup_{i,Z,j} Y = \set{1,2,3,a,b,c,d}/(1 \sim a) = \set{2,3,b,c,d,\spadesuit}
\]
where \(\spadesuit\) is the equivalence class of \(1\) and \(a\); alternatively,
you might think of \(\spadesuit = \{1,a\}\).

\bpoint{Exercise}
Formulate and do the co-Exercise to \ref{fibre-product-exercise}.

\bpoint{Vector Spaces}
Let \(k\) be a field, which you may think of \(k = \mathbf{R}\) to be concrete.
Let \(\mathrm{Vect}\) be the category of finite dimensional vector spaces over
\(k\). Let's think about some properties of this category. Here are a list of
claims that you might enjoy thinking through: Let \(U\), \(V\), and \(W\)
be finite dimensional vector spaces over \(k\).
\begin{enumerate}
\item The initial and terminal object of \(\mathrm{Vect}\) is the vector space
\(\{0\}\).
\item The product and coproduct of \(U\) and \(V\) are given by
\[
U \times V = U \sqcup V \eqqcolon U \oplus V \coloneqq \set{(u,v) | u \in U, v \in V}
\]
where the structure of a vector space on \(U \oplus V\) is given by
\[
(u,v) + (u',v') = (u + u', v + v')
\quad\text{and}\quad
\lambda(u,v) = (\lambda u, \lambda v).
\]
\item Let \(f \colon U \to W\) and \(g \colon V \to W\) be linear maps. Then
\[ U \times_{f,W,g} V = \set{(u,v) \in U \oplus V | f(u) = g(v)}. \]
\item Let \(i \colon U \to V\) and \(j \colon U \to W\) be linear maps. Then
\[ V \sqcup_{i,U,j} W = (V \oplus W)/\set{(i(u), j(u)) | u \in U}. \]
In other words, \(V \sqcup_{i,U,j} W\) is the quotient vector space on \(V \oplus W\)
divided by the image of the map \((i,j) \colon U \to V \oplus W\).
\end{enumerate}

\bpoint{Exercise}
Let \(f \colon U \to W\) be a linear map. Express the \emph{kernel} of \(f\),
\[ \ker(f) \coloneqq \set{u \in U | f(u) = 0} \]
as a fibre product. Likewise, let \(i \colon U \to V\) be a linear map. Express
the \emph{cokernel} of \(i\)
\[ \operatorname{coker}(i) \coloneqq V/\set{i(u) | u \in U} \]
as a pushout.

\bpoint{Tensor Products}
Apparently, the product and coproduct of two vector spaces match, and are
given by the direct sum \(V \oplus W\). Using the relations above, you can
prove that if
\begin{enumerate}
\item \(\set{v_1,\ldots,v_n}\) is a basis for \(V\), and
\item \(\set{w_1,\ldots,w_m}\) is a basis for \(W\),
\end{enumerate}
then \(\set{v_1,\ldots,v_n,w_1,\ldots,w_m}\) is a basis for \(V \oplus W\).
In particular, this means that
\[ \dim(V \oplus W) = \dim(V) + \dim(W) \]
there is an additive relation on the dimension of direct sums. In this sense,
\(V \oplus W\) is thought of as the sum of the vector spaces \(V\) and \(W\).

Then \emph{tensor product} \(V \otimes W\) of \(V\) and \(W\) is another vector
space which is better thought of as the product of \(V\) and \(W\); in
particular, there is a multiplicative relation on dimensions:
\[ \dim(V \otimes W) = \dim(V) \dim(W). \]
One way to define the tensor product is as the vector space\footnote{Meaning that
you are allowed to take \(k\)-linear combinations of the symbols!} on the set
of symbols
\[ \set{v \otimes w | v \in V, w \in W} \]
subject to the following relations: for all \(v, v' \in V\), \(w, w' \in W\),
and \(\lambda \in k\),
\begin{enumerate}
\item \(v \otimes w + v' \otimes w = (v + v') \otimes w\),
\item \(v \otimes w + v \otimes w' = v \otimes (w + w')\),
\item \((\lambda v) \otimes w = \lambda (v \otimes w) = v \otimes (\lambda w)\).
\end{enumerate}
From these, one can prove that a basis for \(V \otimes W\) is given by
\[ \set{v_i \otimes w_j | i = 1,\ldots,n \;\text{and}\; j = 1,\ldots,m } \]
where \(\set{v_1,\ldots,v_n}\) and \(\set{w_1,\ldots,w_m}\) are bases for \(V\)
and \(W\), as above.\footnote{A word of caution: not all elements of
\(V \otimes W\) are of the form \(v \otimes w\). It may well be the case that
an element \(v \otimes w + v' \otimes w'\)  cannot be combined together as
\(v'' \otimes w''\).}

\bpoint{Symmetric Monoidal Categories}
The tensor product operation is \emph{not} internal to the category
\(\mathrm{Vect}\): it is extra \emph{structure} that we can endow
\(\mathrm{Vect}\). The pair \((\mathrm{Vect}, \otimes)\) is the prototype of
a \emph{symmetric monoidal category}. In general, this is a category \(\mathcal{C}\)
together with a binary operation
\[ \otimes \colon \mathcal{C} \times \mathcal{C} \to \mathcal{C} \]
that is ``commutative'' and ``has a unit''. One has to be a bit careful to state
this, since in the setting of categories, you do not necessarily want an \emph{equality}
between \(A \otimes B\) and \(B \otimes A\), but rather an isomorphism. Then there
are conditions that you have to impose on this isomorphism to make it sensible.
See the Wikipedia page \href{https://en.wikipedia.org/wiki/Symmetric_monoidal_category}{here}
for some details, but the short of it is:
a symmetric monoidal category is a category \(\mathcal{C}\) with a operation
\(\otimes\) which allows you to multiply objects together.

\bpoint{Example}
The \(\otimes\) does not need to be really extra structure. For instance,
\(\mathcal{C} = \mathrm{Set}\) with \(\otimes = \times\), the usual Cartesian
product, makes the category of sets into a symmetric monoidal category.

\bpoint{Monoids}
A \emph{monoid} is a set \(M\) with a binary operation \(\mu \colon M \times M \to M\)
such that:
\begin{enumerate}
\item There is a unit: there exists \(e \in M\) such that \(\mu(e,m) = \mu(m, e) = m\)
for all \(m \in M\).
\item It is associative: \(\mu(m, \mu(n, p) = \mu(\mu((m, n), p\) for all
\(m,n,p \in M\).
\end{enumerate}

Now let \((\mathcal{C},\otimes)\) be a (symmetric)\footnote{Symmetry, that is, that
\(\otimes\) is ``commutative'', is not strictly necessary here.} monoidal
category with a terminal object \(\mathbf{1}\). A \emph{monoid object} in
\(\mathcal{C}\) is an object \(\mathbf{M}\) in \(\mathcal{C}\) together with
morphisms
\[ e \colon \mathbf{1} \to \mathbf{M} \quad\text{and}\quad  \mu \colon \mathbf{M} \otimes \mathbf{M} \to \mathbf{M} \]
such that the following two diagrams commute:
\[
\begin{tikzcd}
\mathbf{1} \otimes \mathbf{M} \ar[r,"e \otimes \id_\mathbf{M}"] \ar[dr,"\mathrm{pr}_2"']  & \mathbf{M} \otimes \mathbf{M} \dar["\mu"] & \ar[l,"\id_\mathbf{M} \otimes e"'] \ar[dl,"\pr_1"]  \mathbf{M} \otimes \mathbf{1} \\
& \mathbf{M}
\end{tikzcd}
\quad\text{and}\quad
\begin{tikzcd}
\mathbf{M} \otimes \mathbf{M} \otimes \mathbf{M} \ar[r,"\mu \otimes \id_\mathbf{M}"] \ar[d,"\id_\mathbf{M} \otimes \mu"'] & \mathbf{M} \otimes \mathbf{M} \ar[d,"\mu"] \\
\mathbf{M} \otimes \mathbf{M} \ar[r,"\mu"'] & \mathbf{M}.
\end{tikzcd}
\]
The left diagram is the fact that \(e \colon \mathbf{1} \to M\) behaves like a
unit; and the right diagram is associativity of the multiplication operation
\(\mu\).

\bpoint{Exercise}\label{monoid-exercise}
Let \(\mathbf{M}\) be a monoid object in \(\mathcal{C}\). Let \(X\) be any other object
in \(\mathcal{C}\) and let \(\mathbf{M}(X) \coloneqq \mathcal{C}(X,\mathbf{M})\) be the
set of arrows from \(X\) to \(\mathbf{M}\) in \(\mathcal{C}\). Verify that \(e\) and \(\mu\)
induce maps \(e(X) \colon \mathbf{1}(X) \to \mathbf{M}(X)\) and \(\mu(X) \colon \mathbf{M}(X) \times \mathbf{M}(X) \to \mathbf{M}(X)\)
and show that \((\mathbf{M}(X), e(X), \mu(X))\) is a monoid in the usual sense.\footnote{Hint: \(\mathbf{1}\) is
a terminal object, so there is a only one map from anything to \(\mathbf{1}\).}

\bpoint{Modules and Algebras}
Let \((M,e,\mu)\) be a (plain old) monoid. A \emph{module} over \(M\) is an
set \(P\) together with an action map \(\alpha \colon M \times P \to P\) such that:
For every \(m,n \in M\) and \(p \in P\),
\begin{enumerate}
\item The action is compatible with multiplcation: \(\alpha(\mu(m,n),p) = \alpha(m,\alpha(n,p))\).
\item The identity acts trivially: \(\alpha(e,p) = p\).
\end{enumerate}
An \emph{algebra} over \(M\) is a module \(A\) over \(M\)
together with a binary operation \(\nu \colon A \times A \to A\) such that
for all \(a,b,c \in A\),
it commutes with the action of \(M\):
\[ \nu(\alpha(m,a), \alpha(n,b)) = \alpha(\mu(m,n),\nu(a,b)). \]
We may also ask that \(\nu\) is associative and/or has an identity element, or
place other conditions on \(\nu\).

Now let \((\mathcal{C},\otimes)\) be a symmetric monoidal category with
terminal object \(\mathbf{1}\) and finite products. Let \((\mathbf{M},e,\mu)\)
be a monoid object in \(\mathcal{C}\). A \emph{module} over \(\mathbf{M}\)
is an object \(\mathbf{P}\) of \(\mathcal{C}\) together with a morphism
\(\alpha \colon \mathbf{M} \otimes \mathbf{P} \to \mathbf{P}\) such that the
following diagrams commute:
\[
\begin{tikzcd}
\mathbf{M} \otimes \mathbf{M} \otimes \mathbf{P} \ar[r,"\mu \otimes \id_{\mathbf{P}}"] \ar[d,"\id_{\mathbf{M}} \otimes \alpha"'] &
\mathbf{M} \otimes \mathbf{P} \dar["\alpha"] \\
\mathbf{M} \otimes \mathbf{P} \rar["\alpha"'] & \mathbf{P}
\end{tikzcd}
\quad\text{and}\quad
\begin{tikzcd}
\mathbf{1} \otimes \mathbf{P} \ar[rr,"e \otimes \id_{\mathbf{P}}"] \ar[dr,"\pr_2"'] &&  \mathbf{M} \otimes \mathbf{P} \ar[dl,"\mu"] \\
& \mathbf{P}.
\end{tikzcd}
\]
The left diagram expresses the compatibility between the action map and the multiplication
of the monoid object, and the right expresses the fact that the identity does nothing.

\bpoint{Exercise}
Formulate and repeat Exercise \ref{monoid-exercise} for modules over monoid objects.
Finish the definition of an algebra over a monoid object, and also verify the
corresponding property on morphisms.

\end{document}
||||||| merged common ancestors
=======
\DeclareSymbolFont{AMSb}{U}{msb}{m}{n}
\documentclass[11pt,noamsfonts]{amsart}
\usepackage[left=1.5in, right=1.5in, top=1.2in, bottom=1.2in]{geometry}
\usepackage{mathtools}
\usepackage{braket}
\usepackage{enumitem}
\usepackage[charter,expert]{mathdesign}
\usepackage[scaled=.96,osf]{XCharter}% matches the size used in math
\usepackage[tracking]{microtype}
\usepackage{tikz-cd}
\usepackage{stmaryrd}
\usepackage{comment}
\usepackage[scr=esstix]{mathalfa}

\usepackage{caption}
\usepackage{subcaption}

\usepackage{hyperref}
\linespread{1.04}

\usepackage{enumitem}
\setlist[1]{labelindent=\parindent}
\setlist[enumerate]{labelsep=0.5em}
\setlist[enumerate,1]{label={\upshape (\roman*)}, ref={\upshape (\roman*)}}
\setlist[itemize]{label={--}}

\DeclareMathOperator{\Sym}{Sym}

\makeatletter
\let\c@equation\c@section
\let\theequation\thesection
\makeatother

%found on https://tex.stackexchange.com/a/565122 with improvements from TikZ manual:
\tikzset{>={Straight Barb[length=2pt,width=4pt]}, commutative diagrams/arrow style=tikz}

\usepackage[utf8]{inputenc}

\newcommand{\todo}[1]{\footnote{{\sc\color{red}Todo.} #1}}

% Point Formats
\newcommand{\pointheader}{\vspace{2mm}\noindent\refstepcounter{section}\textbf{\thesection.}}
\newcommand{\point}{\pointheader~}
\newcommand{\tpoint}[1]{\pointheader~{\bf #1. ---}}
\newcommand{\epoint}[1]{\pointheader~{\em #1.}}
\newcommand{\bpoint}[1]{\pointheader~{\bf #1.}}

% QED Symbol
\newcommand{\psqedsymb}{\(\blacksquare\)}
\renewcommand{\qed}{~\hfill{\psqedsymb}}


\makeatletter
\newcommand*{\coloneqq}{\mathrel{\rlap{%
           \raisebox{0.3ex}{$\m@th\cdot$}}%
           \raisebox{-0.3ex}{$\m@th\cdot$}}%
           =}
\newcommand{\eqqcolon}{=%
           \mathrel{\rlap{%
           \raisebox{0.3ex}{$\m@th\cdot$}}%
           \raisebox{-0.3ex}{$\m@th\cdot$}}}
\makeatother

\DeclareMathOperator{\Hom}{Hom}

\title{Notes for July 8, 2021}
\begin{document}
\maketitle

\bpoint{Binomial Coefficients}
Let \(n\) and \(k\) be positive integers. Then the binomial coefficient
\[
\binom{n}{k} \coloneqq \frac{n!}{k!(n-k)!}
\]
has many combinatorial interpretations including:
\begin{enumerate}
\item it is the number of binary strings of length \(n\) with exactly
\(k\) appearances of \(1\);
\item it is the number of Dyck paths of length \(n\) with exactly \(k\)
upward steps;
\item it is the number of size \(k\) subsets of a size \(n\) set.
\end{enumerate}
One categorification of the binomial coefficient in the category of sets then
might be:
\begin{align*}
\binom{-}{-} \colon \mathrm{Set} \times \mathrm{Set} \to \mathrm{Set}
(N,K) & \mapsto \set{S \in \Hom(N,[2]) | S\;\text{is isomorphic to}\;K}.
\end{align*}
In words, this is the functor that takes a pair of finite sets \((N,K)\),
produces the set \(\Hom(N,[2])\) of all subsets of \(N\)---binary strings
correspond to subsets of \(N\)!---and then picks out the subsets with the same
size as \(K\).\footnote{Note, however, that in order to make this into an
honest functor, one might have to be careful about what morphisms to take in
the categories of sets in question.} This is a categorification of the binomial
coefficient in the sense that the following diagram commutes:
\[
\begin{tikzcd}
\mathrm{Set} \times \mathrm{Set} \ar[r,"\binom{-}{-}"] \ar[d,"\#"'] & \mathrm{Set} \ar[d,"\#"] \\
\mathbf{Z}_+ \times \mathbf{Z}_+ \ar[r,"\binom{-}{-}"] & \mathbf{Z}_+
\end{tikzcd}
\]
which means, explicitly, given finite sets \(N\) and \(K\) with sizes \(n \coloneqq \#N\)
and \(k \coloneqq \#K\), there is the relation
\[ \#\binom{N}{K} = \binom{\#N}{\#K} = \binom{n}{k}. \]

This might be carried out in other categories, too! The only thing that was
special to the category of sets was the construction of the set of all subsets
of \(N\). But this amounted to saying that the object \([2]\) had the property
that \(\Hom(N,[2])\) classified subobjects of \(N\): in other words, \([2]\)
is a \emph{subobject classifier} in the category of sets. So we might
perform the same construction in a general category \(\mathsf{C}\) by replacing
\([2]\) in the above formula with a subobject classifier \(\Omega\) in \(\mathsf{C}\).

\bpoint{Representations of Finite Groups}
Let \(G\) be a group. A \emph{representation of \(G\)} on a finite-dimensional
complex vector space \(V\) is a linear action of \(G\) on \(V\): that is, this
is a binary operator \(\cdot \colon G \times V \to V\)
such that for every \(g, h \in G\), \(v,w \in V\), and \(\lambda \in \mathbf{C}\),
\begin{enumerate}
\item \(g \cdot (v + \lambda w) = (g \cdot v) + \lambda (g \cdot w)\),
\item \(h \cdot (g \cdot v) = (hg) \cdot v\), and
\item \(e \cdot v = v\),
\end{enumerate}
where \(e \in G\) is the identity element. Equivalently, let
\[ \mathrm{GL}(V) \coloneqq \set{T \colon V \to V | T\;\text{an invertible linear transformation}}, \]
then the data of a representation of \(G\) on \(V\) amounts to a
homomorphism \(\rho \colon G \to \mathrm{GL}(V)\).\footnote{Note that when \(V =
\mathbf{C}^{\oplus n}\) is the standard \(n\)-dimensional vector space, then
\(\mathrm{GL}(V) = \mathrm{GL}_n(\mathbf{C})\) is none other than the group of
invertible square complex matrices of size \(n\): this is the correspondence
between linear transformations and matrices, upon picking a basis.} A subspace
\(W \subseteq V\) is a \emph{subrepresentation} if
\[ g \cdot w \in W \quad\text{for all}\;g \in G\;\text{and all}\; w \in W. \]
A representation \(V\) is called \emph{irreducible} if whenever \(W \subseteq V\)
is a subrepresentation, then either \(W = \{0\}\) or \(W = V\).

The basic problem of representation theory is to classify irreducible
representations of finite groups \(G\), the point being that irreducible
representations are the building blocks of all representations. Over the complex
numbers, all representations are built out of irreducible ones in the simplest
possible way:

\tpoint{Maschke's Theorem}
\emph{Let \(G\) be a finite group. Then for every complex representation \(V\),
there is a decomposition
\[ V \cong \bigoplus\nolimits_{i \in I} V_i \]
where \(I\) is a finite index set, and the \(V_i\) are irreducible
representations of \(G\).}

\point
Moreover, the essential data of an irreducible representation turns out to be
something quite intrinsic to \(G\). Namely, let \(V\) be a representation of
\(G\), with associated homomorphism \(\rho \colon G \to \mathrm{GL}(V)\).
The \emph{character} of this representation is the homomorphism
\[ \chi \coloneqq \mathrm{trace} \circ \rho \colon G \to \mathrm{GL}(V) \to \mathbf{C} \]
obtained by taking an element \(g \in G\), considering the linear
transformation \(\rho(g) \in \mathrm{GL}(V)\), choosing any basis of \(V\) to
make \(\rho(g)\) into a complex matrix, and then taking the trace of the
resulting matrix. The properties of the trace ensure that this is independent
of the choice of basis chosen, and hence this is really an invariant of the
representation \(\rho\) itself. Then the classification theorem for irreducible
representations is

\tpoint{Character Theory of Finite Groups}
\emph{Let \(G\) be a finite group. Let \(V\) and \(W\) be irreducible
representations of \(G\) with characters \(\chi\) and \(\psi\), respectively.
Then \(V\) and \(W\) are isomorphic representations if and only if \(\chi\) and
\(\psi\) are the same function on \(G\).}

\point
Thus the problem of classifying irreducible representations of \(G\) is greatly
aided by the characters. Here are a few more facts:
\begin{enumerate}
\item characters are \emph{class functions}, that is, if \(g, h \in G\)
are in the same \href{https://en.wikipedia.org/wiki/Conjugacy_class}{conjugacy
class}\footnote{Read: if up to a change of labels/coordinates, they are the
same.}, then \(\chi(g) = \chi(h)\);
\item the set of all complex-valued class functions out of \(G\) is a
vector space with dimension equal to the number of conjugacy classes of \(G\):
take the indicator function for each conjugacy class;
\item there is an inner product on the space of class functions;
\item the irreducible characters form an orthogonal basis for the space of
class functions;
\item each irreducible representations of \(G\) is a direct summand, appearing
exactly once, of the \emph{regular representation} of \(G\):
\[ \mathbf{C}G \coloneqq \Set{\sum\nolimits_{g \in G} a_g g | a_g \in \mathbf{G}} \]
this is the vector space spanned by symbols indexed by elements of \(G\) and
\(G\) acts on this by multiplication on the symbols.
\end{enumerate}

\bpoint{Young Diagrams, Tableaux}
A \href{https://en.wikipedia.org/wiki/Young_tableau}{\emph{Young diagram}} is a
set of boxes that is justified in some fashion. There are many conventions, but
let's use the so-called English convention in which the boxes are left and upper
aligned. The number of boxes in each row of a Young diagram \(\mathsf{Y}\)
form a partition  \(\lambda = (\lambda_1 \geq \lambda_2 \geq \lambda_3 \geq \ldots)\),
that is, a non-increasing sequence of nonnegative integers. The relevance of
Young diagrams appears in their connection with the representation theory of
the symmetric group: the set of irreducible representations of the symmetric
group \(S_n\) on \(n\) elements are naturally indexed by a Young diagram with \(n\)
boxes. In fact, given a Young diagram \(\mathsf{Y}\) with \(n\) boxes, the
corresponding \href{https://en.wikipedia.org/wiki/Specht_module}{irreducible
representation} can be made by making a vector space whose basis consists of
certain fillings of \(\mathsf{Y}\).

Another related thing involves \href{https://en.wikipedia.org/wiki/Symmetric_function}{symmetric functions}.
There are certain symmetric polynomials called the
\href{https://en.wikipedia.org/wiki/Schur_polynomial}{Schur polynomials}, and
they can be specified by taking and counting certain Young diagrams. These are
related to the irreducible polynomial representations of the general linear
group \(\mathrm{GL}_n(\mathbf{C})\), which in turn is related to the symmetric
group situation via what is known as \href{https://en.wikipedia.org/wiki/Schur%E2%80%93Weyl_duality}{Schur--Weyl duality}.
Everything is connected!

\bpoint{Universal Invariants}
Finally we discussed the notion of finding some sort of universal way of making
an algebraic invariant for some type of object, satisfying some set of
relations. The one that I know best is the notion of an \emph{Euler characteristic}.
The basic example would be: I am looking for an invariant \(\chi\) on the set
\(\mathrm{Vect}\) of finite-dimensional vector spaces such that
if I can write \(V = V_1 \oplus V_2\), then
\[ \chi(V) = \chi(V_1) + \chi(V_2). \]
Well, what sort of structure do I need to make sense of this? I just need to
be able to add my invariants, so I am probably looking for a function
\[ \chi \colon \mathrm{Vect} \to G \]
for some abelian group \(G\). I want this to be \emph{universal} in the sense
that if I have any other map \(\psi \colon \mathrm{Vect} \to H\) to some
other abelian group \(H\), then I would like there to be a unique homomorphism
\(\bar\psi \colon G \to H\) such that the following diagram commutes:
\[
\begin{tikzcd}
\mathrm{Vect}\rar["\chi"] \ar[dr,"\psi"']  & G \ar[d,"\bar\psi"] \\ & H
\end{tikzcd}
\quad\text{that is,}\; \psi = \bar\psi \circ \chi.
\]
To prove that this exists, I need to construct such an object. And the right
thing to do is the least lossy thing possible: take \(G\) to be the free
abelian group generated on isomorphism classes of objects on \(\mathrm{Vect}\),
subject to the additive relations. Explicitly, the isomorphism classes of objects
in \(\mathrm{Vect}\) are represented by the standard vector spaces \(\mathbf{C}^n\)
of dimension \(n\), for each integer \(n \geq 0\). Then I would like to set \(G\) to be
the abelian group generated by symbols \([\mathbf{C}^n]\). But I need to
subject these to the direct sum relation. Note, however, that
\[
\mathbf{C}^n
= \mathbf{C}^{n-1} \oplus \mathbf{C}
= \cdots \mathbf{C} \oplus \mathbf{C} \oplus \cdots \oplus \mathbf{C}
\]
by peeling off coordinates. So this means that I need to impose the relations
\[ [\mathbf{C}^n] = [\mathbf{C}] + [\mathbf{C}] + \cdots + [\mathbf{C}] = n[\mathbf{C}]. \]
Hence \(G\) is the free abelian group generated on the single symbol \([\mathbf{C}]\), and hence
is the group of integers \(G \cong \mathbf{Z} \cdot [\mathbf{C}]\)
and the function \(\chi \colon \mathrm{Vect} \to G\) is none other than the
dimension function!

\point
This construction can be repeated when the category of vector spaces is replaced
by any other category \(\mathcal{A}\) in which some notion of ``an object is
the sum of two others'' is represent. This is axiomitized in the notion of a
\href{https://en.wikipedia.org/wiki/Abelian_category}{abelian category}
and the resulting construction is known as the
\href{https://en.wikipedia.org/wiki/Grothendieck_group}{Grothendieck group} of
the associated category.



\end{document}
