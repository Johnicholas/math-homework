In elementary category theory, we learn about initial and final objects.

In some categories (such as vector spaces) the initial object and the final object are isomorphic.
The collection of all categories where there is an initial object, a final object, and they are isomorphic is called a "pointed category".

There is one linear map from the one-point vector space to any other vector space (zero goes to zero).
There is one linear map from any vector space to the one-point vector space (everything goes to zero).

There is an idea of "exactness" of a chain of morphisms; I am not going to say the definition of exactness, because I don't understand it well enough yet.

The way I understand it, there are powerful ideas in Set
"injective", "surjective", and "bijective".

You can say something like "injective" by saying "0 \to X \to Y is exact".
You can say something like "surjective" by saying "X \to Y \to 0 is exact".
You can say something like "bijective" by saying "0 \to X \to Y \to 0 is exact".

This motivates asking - what does "0 \to X \to Y \to Z \to 0" mean about the triple X, Y, and Z.

The way I understand it, an example of this in the category of vector spaces is:
1. The first map from 0 to X might be taking the zero space to the origin, considered as a subspace of the y axis.
2. The second map from X to Y might be including the y axis into the whole two-dimensional plane. (injective)
3. The third map from Y to Z might be collapsing the whole two-dimensional plane to the x axis. (surjective)
4. The fourth map from Z to 0 might be collapsing the whole x axis back to 0.

This concrete example convinces me that maybe an exact sequence 0 \to X \to Y to Z \to 0 is somehow related to viewing Y as some sort of product or sum X \oplus Z.

In the contraction-deletion view of the Tutte invariant,
there is a skein relation, something like:

f(G)=f(G\setminus e)+f(G/e)

G \setminus e has an inclusion into G
G has a surjection onto G / e

Is it true that if we have three graphs connected by an injection / surjection pair of graph morphisms,
then the Tutte invariants of the three graphs will have the corresponding F(Y) = F(X) + F(G) structure?

I have heard (reading wikipedia) that if G is the union of disjoint graphs H and H' then T(G) = T(H) * T(H')

If H and H' were identical, then there would be an injection from H to G, to one of the two copies,
and a surjection from G to H, a double cover.
So T(G) is probably not always equal to T(H) + T(H)?

2 x = x * x implies x = 2
if T(H) were 2, then T would be a graph with
