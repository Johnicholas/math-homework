\documentclass[11pt, noamsfonts]{amsart}
\usepackage[left=1.5in, right=1.5in, top=1.2in, bottom=1.2in]{geometry}
\usepackage[charter,expert]{mathdesign}
\usepackage[scaled=.96,osf]{XCharter}% matches the size used in math
\usepackage{hyperref}

\usepackage{tikz-cd}
\tikzset{>={Straight Barb[length=2pt,width=4pt]}, commutative diagrams/arrow style=tikz}

\linespread{1.04}

\usepackage{enumitem}
\setlist[1]{labelindent=\parindent}
\setlist[enumerate]{labelsep=0.5em}
\setlist[enumerate,1]{label={\upshape (\roman*)}, ref={\upshape (\roman*)}}
\setlist[itemize]{label={--}}

% Point Formats
\newcommand{\pointheader}{\vspace{2mm}\noindent\refstepcounter{section}\textbf{\thesection.}}
\newcommand{\point}{\pointheader~}
\newcommand{\tpoint}[1]{\pointheader~{\bf #1. ---}}
\newcommand{\epoint}[1]{\pointheader~{\em #1.}}
\newcommand{\bpoint}[1]{\pointheader~{\bf #1.}}

% QED Symbol
\newcommand{\psqedsymb}{\(\blacksquare\)}
\renewcommand{\qed}{~\hfill{\psqedsymb}}


\makeatletter
\newcommand*{\coloneqq}{\mathrel{\rlap{%
           \raisebox{0.3ex}{$\m@th\cdot$}}%
           \raisebox{-0.3ex}{$\m@th\cdot$}}%
           =}
\newcommand{\eqqcolon}{=%
           \mathrel{\rlap{%
           \raisebox{0.3ex}{$\m@th\cdot$}}%
           \raisebox{-0.3ex}{$\m@th\cdot$}}}
\makeatother

\title{Notes for July 22, 2021}
\begin{document}
\maketitle

\bpoint{Objects and Functors}
Let \(\mathcal{C}\) be a category and \(f \colon X \to Y\) be an arrow between
two objects of \(\mathcal{C}\). Let \(Z\) be any other object of \(\mathcal{C}\).
Then there is an induced map of sets
\[ f_{*,Z} \colon \mathcal{C}(Z,X) \to \mathcal{C}(Z,Y) \]
from the set of arrows from \(Z\) to \(X\), to that from \(Z\) to \(Y\). This,
of course, can be described directly: given an arrow \(g \colon Z \to X\),
set \(f_{*,Z}(g) \coloneqq f \circ g \colon Z \to X \to Y\).

There are two ways that this can be packaged conceptually. A first is to think
of \(f_{*,Z}\) as the result of applying the functor
\[ \mathcal{C}(Z,-) \colon \mathcal{C} \to \mathrm{Set} \]
which takes objects of \(\mathcal{C}\) to the set of arrows it receives from
\(Z\). In this case, \(f_{*,Z}\) is perhaps viewed as part of the
\emph{construction} of the functor \(\mathcal{C}(Z,-)\).

A second way to think of \(f_{*,Z}\) is that it defines a natural transformation
between the two representable functors
\[
\mathcal{C}(-,X) \colon \mathcal{C}^{\mathrm{opp}} \to \mathrm{Set}
\quad\text{and}\quad
\mathcal{C}(-,Y) \colon \mathcal{C}^{\mathrm{opp}} \to \mathrm{Set}.
\]
That is: \(f_* \colon \mathcal{C}(-,X) \to \mathcal{C}(-,Y)\) is the natural
transformation such that its component \(f_{*,Z}\) on an object \(Z\) of
\(\mathcal{C}\) is described as above.

\bpoint{Monoid Objects, Revisited}
Let \((\mathcal{C},\otimes)\) be a symmetric monoidal category, and let
\((\mathbf{M}, e, \mu)\) be a monoid object therein; recall that this consists
of an object \(\mathbf{M}\) together with two morphisms
\[
e \colon \mathbf{1} \to \mathbf{M}
\quad\text{and}\quad
\mu \colon \mathbf{M} \otimes \mathbf{M} \to \mathbf{M}
\]
in \(\mathcal{C}\) satisfying some relations that may be expressed in diagrams,
see notes from last week. Let \(X\) be any object of \(\mathcal{C}\). The
discussion above induces maps of sets
\[
e_{*,X} \colon \mathbf{1}(X) \to \mathbf{M}(X)
\quad\text{and}\quad
\mu_{*,X} \colon (\mathbf{M} \otimes \mathbf{M})(X) \to \mathbf{M}(X)
\]
where \(\mathbf{1}(X) \coloneqq \mathcal{C}(X,\mathbf{1})\),
\(\mathbf{M}(X) \coloneqq \mathcal{C}(X,\mathbf{M})\), and
\((\mathbf{M} \otimes \mathbf{M})(X) \coloneqq \mathcal{C}(X,\mathbf{M} \otimes \mathbf{M})\).
This almost gives the set \(\mathbf{M}(X)\) a binary operation, but
\((\mathbf{M} \otimes \mathbf{M})(X)\) might be a bit different from
\(\mathbf{M}(X) \times \mathbf{M}(X)\) as sets. Nonetheless, there is a
canonical map from the latter to the former. The relations for \(e\)
and \(\mu\) then carry over to relations for \(e_{*,X}\) and \(\mu_{*,X}\)
that translate to the usual monoid relations.

\bpoint{Extending Diagrams}
Many categorical constructions are of the following nature: Fix your favourite
category \(\mathcal{C}\). Let \(\mathcal{D}\) be a category which is thought
of as a diagram expressing arrows between objects. Let
\(\mathrm{draw} \colon \mathcal{D} \to \mathcal{C}\) be a functor, which might
be thought of as drawing the diagram \(\mathcal{D}\) in \(\mathcal{C}\). Then
a question one might ask would be: can \(\mathrm{draw}(\mathcal{D})\) be
extended to some (fixed) bigger diagram? For example, consider the following
augmentations of \(\mathcal{D}\):
\begin{enumerate}
\item \(\mathcal{D}_\vee\) is \(\mathcal{D}\) together with a new terminal
object, maybe thought of as an object below, receiving a unique map from everything else.
\item \(\mathcal{D}^\wedge\) is \(\mathcal{D}\) together with a new initial
object, maybe thought of as an object on top, giving a unique map to everything else.
\item \(\mathcal{D}(\star)\) is \(\mathcal{D}\) together with a new object
giving and receiving \(5\) distinct maps to every other object in \(\mathcal{D}\).
\end{enumerate}
Generally, this might be a situation in which we have another category
\(\tilde{\mathcal{D}}\) together with a functor
\(\iota \colon \mathcal{D} \to \tilde{\mathcal{D}}\), perhaps be assumed to be
faithful and injective on objects for sanity's sake. Then we are asking for all
ways to fill in the dashed arrow
\[
\begin{tikzcd}
\mathcal{D} \rar["\mathrm{draw}"] \dar["\iota"'] & \mathcal{C} \\
\tilde{\mathcal{D}} \ar[dashed,ur]
\end{tikzcd}
\]
so that the diagram commutes, perhaps up to natural isomorphisms. The first
question to ask is: does there \emph{exist at least one} dashed arrow? Then, if
so, is there a \emph{best} dashed arrow amongst all possible choices?

To make sense of the notion of best, consider all possible ways of filling
in the dashed arrow. One way to do this is to consider category
\([\tilde{\mathcal{D}}, \mathcal{C}]\) of all functors from \(\tilde{\mathcal{D}}\)
to \(\mathcal{C}\). Therein, we would like to consider the full subcategory of
all possible dashed arrows:
\[
[\tilde{\mathcal{D}}, \mathcal{C}]_{\mathrm{draw}} \coloneqq
\{F \colon \tilde{\mathcal{D}} \to \mathcal{C} \mid \mathrm{draw} \cong F \circ \iota\}.
\]
Then our questions above amount to asking whether or not this category is
nonempty, and, if so, whether or not there is either a terminal or initial
object.

To relate this with some concrete notions, consider the examples above:
\begin{enumerate}
\item A terminal object in \([\mathcal{D}_\vee,\mathcal{C}]_{\mathrm{draw}}\)
is a \emph{colimit} of \(\mathcal{D}\) in \(\mathcal{C}\).
\item An initial object in \([\mathcal{D}^\wedge,\mathcal{C}]_{\mathrm{draw}}\)
is a \emph{limit} of \(\mathcal{D}\) in \(\mathcal{C}\).
\end{enumerate}
(The way I remember which is which is to imagine products and coproducts. Then
a product is a limit and a coproduct is a colimit.)

\bpoint{Abelian groups}
Consider the category \(\mathrm{Ab}\) of abelian groups. This is a category with
a lot of structure. For instance, the zero abelian group \(\{0\}\) is the initial
and terminal object of the category. For any pair of abelian groups \(A\) and
\(B\), the product \(A \times B\) and coproduct \(A \sqcup B\) coincide and
are given by the direct sum
\[ A \times B = A \sqcup B = A \oplus B \coloneqq \{(a,b) \mid a \in A, b \in B\}, \]
the group structure being pointwise addition. Thus, despite their names as
products, it is helpful to think of this structure as yielding a way of ``adding
two abelain groups'' to make a bigger abelian group.

To continue below, recall also the following: For any homomorphism
\(f \colon A \to B\) of abelian group, the categorical
\href{https://en.wikipedia.org/wiki/Kernel_(category_theory)}{\emph{kernel}} and
\href{https://en.wikipedia.org/wiki/Cokernel}{\emph{cokernel}} exist
and are given by
\[
\ker(f) \coloneqq \{a \in A \mid f(a) = 0\}
\quad\text{and}\quad
\operatorname{coker}(f) \coloneqq B/f(A)
\]
the latter being the quotient abelian group.

But there are perhaps other ways to combine abelian groups. Namely, given an
injection of abelian groups \(f \colon A \to B\), we might think of \(A\) as being ``smaller''
than \(B\). The difference is then measured by the quotient group \(C \coloneqq B/A\);
this has a surjection \(g \colon B \to C\), so we might think of \(B\) as ``bigger'' than \(C\).
In a way, \(B\) is then the sum of the pieces \(A\) and \(B\), and we may express
this idea by writing
\[ 0 \to A \xrightarrow{f} B \xrightarrow{g} C \to 0 \]
and saying that this is an \emph{exact sequence} of abelian groups. In general,
given three abelian groups \(A\), \(B\), and \(C\), and homomorphisms \(f \colon A \to B\)
and \(g \colon B \to C\), say that the sequence above is a \emph{short exact sequence}
if
\begin{enumerate}
\item \(f\) is injective,
\item \(g\) is surjective, and
\item \(\ker(g) = \operatorname{coker}(f)\).
\end{enumerate}
These conditions express the fact that \(A\) is a subgroup of \(B\), \(C\) is a
quotient group of \(B\), and that the subgroup of \(B\) you are dividing out by
to get \(C\) is \(A\).

For a simple example, consider the direct sum \(B = A \oplus C\) of two groups
\(A\) and \(C\). Let \(i \colon A \to A \oplus C\) be the inclusion into the first factor
and \(p \colon A \oplus C \to C\) the projection onto the second factor. Then these maps
fit into a short exact sequence
\[ 0 \to A \xrightarrow{i} A \oplus C \xrightarrow{p} C \to 0. \]
Such short exact sequences are called \emph{split} sequences. In general, a
short exact sequence is said to be \emph{split} if there exists a commutative
diagram of short exact sequences
\[
\begin{tikzcd}
0 \rar & A \rar["f"] \dar & B \rar["g"] \dar & C \dar \rar & 0 \\
0 \rar & A \rar["i"] & A \oplus C \rar["p"] & C \rar & 0
\end{tikzcd}
\]
such that each of the vertical maps are isomorphisms.

But not all short exact sequences are of this form in \(\mathrm{Ab}\): for example,
consider \(A = \mathbf{Z}\) \(B = \mathbf{Z}\), and \(f \colon \mathbf{Z} \to \mathbf{Z}\)
given by multiplication by \(2\). Then, setting \(C = \mathbf{Z}/2\mathbf{Z}\)
and \(g \colon \mathbf{Z} \to \mathbf{Z}/2\mathbf{Z}\) the reduction map, we
have a short exact sequence
\[ 0 \to \mathbf{Z} \xrightarrow{\times 2} \mathbf{Z} \to \mathbf{Z}/2\mathbf{Z} \to 0 \]
and \(\mathbf{Z}\) is certainly not isomorphic to \(\mathbf{Z} \oplus \mathbf{Z}/2\mathbf{Z}\):
for instance, there is no nonzero element \(x\) of \(\mathbf{Z}\) such that \(x + x = 0\).
In this case, we say that the short exact sequence is \emph{not split}.

\bpoint{Exercise}
Prove the ``Splitting Lemma'': given a short exact sequence
\[ 0 \to A \xrightarrow{f} B \xrightarrow{g} C \to 0 \]
the following are equivalent:
\begin{enumerate}
\item the short exact sequence is split,
\item there exists a homomorphism \(s \colon C \to B\) such that \(g \circ s = \operatorname{id}_C\), and
\item there exists a homomorphism \(r \colon B \to A\) such that \(r \circ f = \operatorname{id}_A\).
\end{enumerate}

\bpoint{Enriched Category}
Let \(\mathcal{C}\) be a category. Then taking the set of arrows between objects
might be viewed as a functor
\[ \mathcal{C}(-,-) \colon \mathcal{C}^{\mathrm{opp}} \times \mathcal{C} \to \mathrm{Set}. \]
One might ask whether or not this can be lifted to a functor with values in a
category \(\mathcal{A}\) other than \(\mathrm{Set}\). If this is possible, then
we say that \(\mathcal{C}\) is \emph{enriched in \(\mathcal{A}\)}.

For instance, when \(\mathcal{C} = \mathrm{Ab}\) is the category of abelian
groups, then the sets \(\mathrm{Ab}(A,B)\) of homomorphism between two groups
\(A\) and \(B\) is itself an abelian group under pointwise addition, and
unit coming from the zero morphism. Likewise, when \(\mathcal{C} = \mathrm{Vect}\)
is the category of vector spaces, the sets \(\mathrm{Vect}(V,W)\) of linear maps
between two vector space is itself a vector space under pointwise addition and
scaling.

\bpoint{Tensor Product Universally}
Consider the category \(\mathrm{Vect}\) of finite dimensional vector spaces
over a field \(k\). The tensor product furnishes \(\mathrm{Vect}\) with a
functor
\[ \otimes \colon \mathrm{Vect} \times \mathrm{Vect} \to \mathrm{Vect}. \]
This functor is universal in some sense, most easily made clear as follows.
Let \(U\) and \(V\) be a pair of vector spaces. Then for every vector space
\(W\) and \emph{bilinear} map \(b \colon U \times V \to W\)\footnote{This means that
\(b(\lambda u + u', v) = \lambda b(u,v) + b(u',v)\) and \(b(u,\lambda v + v') = \lambda b(u,v) + b(u,v')\)
for every \(u,u' \in U\), \(v,v' \in V\), and \(\lambda \in k\).}, there exists
a unique \emph{linear} map \(\tilde b \colon U \otimes V \to W\) such that the diagram
\[
\begin{tikzcd}
U \times V \rar \ar[dr,"b"'] & U \otimes V \dar["\tilde b"] \\ & W
\end{tikzcd}
\]
commutes. Here, the map \(U \times V \to U \otimes V\) is the bilinear map
\((u,v) \mapsto u \otimes v\). In other words, the tensor product is the
universal way to transform \emph{bilinear} maps out of two vector spaces to a
\emph{linear} map out of a single vector space.

\bpoint{Hom-Tensor Adjunction}
Let \(U\), \(V\), and \(W\) be finite-dimensional vector spaces. Then there is
a canonical isomorphism
\[ \mathrm{Vect}(U \otimes V, W) \cong \mathrm{Vect}(U, \mathrm{Vect}(V,W)). \]
Here, we view \(\mathrm{Vect}(V,W)\) as a vector space as explained above.
One way to see this is to use the universal property above: \(\mathrm{Vect}(U \otimes V,W)\)
is the set of bilinear maps \(U \times V \to W\). But by partially evaluating a
bilinear map in the first argument, I obtain a linear map \(V \to W\). This
proves the equality.

Then what the above expresses is that the two functors
\[
- \otimes V \colon \mathrm{Vect} \to \mathrm{Vect}
\quad\text{and}\quad
\mathrm{Vect}(V,-) \colon \mathrm{Vect} \to \mathrm{Vect}
\]
are adjoint to one another, with \(-\otimes V\) the left adjoint (because it
is on the left side of the set of maps), and \(\mathrm{Vect}(V,-)\) the right
adjoint. This structure may be abstracted to a general symmetric monoidal category,
yielding the notion of an \href{https://ncatlab.org/nlab/show/internal+hom}{\emph{internal hom}}
and a \href{https://ncatlab.org/nlab/show/closed+monoidal+category}{\emph{closed monoidal category}}.
\end{document}
