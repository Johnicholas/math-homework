\DeclareSymbolFont{AMSb}{U}{msb}{m}{n}
\documentclass[11pt,noamsfonts]{amsart}
\usepackage[left=1.5in, right=1.5in, top=1.2in, bottom=1.2in]{geometry}
\usepackage{mathtools}
\usepackage{braket}
\usepackage{enumitem}
\usepackage[charter,expert]{mathdesign}
\usepackage[scaled=.96,osf]{XCharter}% matches the size used in math
\usepackage[tracking]{microtype}
\usepackage{tikz-cd}
\usepackage{stmaryrd}
\usepackage{comment}
\usepackage[scr=esstix]{mathalfa}

\usepackage{caption}
\usepackage{subcaption}

\usepackage{hyperref}
\linespread{1.04}

\usepackage{enumitem}
\setlist[1]{labelindent=\parindent}
\setlist[enumerate]{labelsep=0.5em}
\setlist[enumerate,1]{label={\upshape (\roman*)}, ref={\upshape (\roman*)}}
\setlist[itemize]{label={--}}

\DeclareMathOperator{\Sym}{Sym}

\makeatletter
\let\c@equation\c@section
\let\theequation\thesection
\makeatother

%found on https://tex.stackexchange.com/a/565122 with improvements from TikZ manual:
\tikzset{>={Straight Barb[length=2pt,width=4pt]}, commutative diagrams/arrow style=tikz}

\usepackage[utf8]{inputenc}

\newcommand{\todo}[1]{\footnote{{\sc\color{red}Todo.} #1}}

% Point Formats
\newcommand{\pointheader}{\vspace{2mm}\noindent\refstepcounter{section}\textbf{\thesection.}}
\newcommand{\point}{\pointheader~}
\newcommand{\tpoint}[1]{\pointheader~{\bf #1. ---}}
\newcommand{\epoint}[1]{\pointheader~{\em #1.}}
\newcommand{\bpoint}[1]{\pointheader~{\bf #1.}}

% QED Symbol
\newcommand{\psqedsymb}{\(\blacksquare\)}
\renewcommand{\qed}{~\hfill{\psqedsymb}}


\makeatletter
\newcommand*{\coloneqq}{\mathrel{\rlap{%
           \raisebox{0.3ex}{$\m@th\cdot$}}%
           \raisebox{-0.3ex}{$\m@th\cdot$}}%
           =}
\newcommand{\eqqcolon}{=%
           \mathrel{\rlap{%
           \raisebox{0.3ex}{$\m@th\cdot$}}%
           \raisebox{-0.3ex}{$\m@th\cdot$}}}
\makeatother

\DeclareMathOperator{\Hom}{Hom}

\title{The Eckmann-Hilton argument}
\begin{document}
\maketitle


"The Eckmann-Hilton argument, higher operads and En-spaces." by M. Batanin starts by saying "The classical Eckmann-Hilton argument shows that two monoid structures on a set, such that one is a homomorphism for the other, coincide and, moreover, the resulting monoid is commutative."

TODO: work on understanding this:
\url{https://link.springer.com/content/pdf/10.1007/BF01451367.pdf}

\tpoint{6.1 of "Group-like Structures":}
\label{6.1}

\tpoint{For any \(D \in \mathfrak{D}, A \in \mathfrak{C}, f_j : S A \to D\) in \(\mathfrak{D},  j = \{ 1, 2 \} \),}
\begin{align}
\label{6.6}
\eta \{ f_1, f_2 \} = \{ \eta f_1, \eta f_2 \}
\end{align}

\tpoint{\( \eta \) is isomorphism of \( \underline{M} \)-sets}
\label{6.7}

\begin{proof}
\begin{align*}
\eta ( \mu \circ \{ f_1, f_2 \} ) = & T \mu \circ \eta \{ f_1 , f_2 \} \tag{By \ref{6.1}} \\
= & T \mu \circ \{ \eta f_1 , \eta f_2 \} \tag{By \ref{6.6}}
\end{align*}
\end{proof}



\tpoint{Theorem 6.8 of "Group-like Structures" says: (Eckman-Hilton)}
Let \( m_T\) be a natural family of \(\underline{H}\)-structures in \(\mathfrak{C}\) and let \(\mu\) be an \(\underline{H}\)-structure. Under these hypotheses \(m_TB = T \mu\) and is commutative.

\begin{proof}
Let \(A\) be a fixed but arbitrary object of \(\mathfrak{C}\). Then \( H ( S A , B ) \) receives structure either from the \(\overline{H}\)-structure \( n_{S A} \) given by \ref{6.3} or from the structure \( \mu \) in \( B \). By Theorem \ref{4.17} these two \( \underline{H} \)-structures coincide in \( H( S A , B ) \) 

On the other hand \( \eta : H( S A, B ) \to H (A , T B) \) is an isomorphism of \( \underline{H} \) -sets either if we use the structures \( n_{S A} \) and \( m_{T B} \) (Theorem \ref{6.4} ) or if we use the structure \( \mu \) and \(T \mu \) (Prop \ref{6.7}).

It follows that the structures \( m_{T B} \) and \(T \mu\) induce the \emph{same commutative} structure in \( H(A, T B) \). Since this is true for every \( A \)  in \( \mathfrak{C} \), we must have \( m_{T B} = T \mu \) and each is commutative.

\end{proof}


UGH, what does this mean.

\end{document}
