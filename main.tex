\documentclass{proc-l}
\usepackage[utf8]{inputenc}

\newtheorem{theorem}{Theorem}[section]
\newtheorem{lemma}[theorem]{Lemma}

\theoremstyle{definition}
\newtheorem{definition}[theorem]{Definition}
\newtheorem{example}[theorem]{Example}
\newtheorem{xca}[theorem]{Exercise}

\theoremstyle{remark}
\newtheorem{remark}[theorem]{Remark}

\numberwithin{equation}{section}



\title{math-homework}
\author{Johnicholas Hines}
\date{July 2021}

\begin{document}

\maketitle

\section{Products are products and substituting an isomorphic object leads to an equivalent category}

%Consider the rule
%\[
%  \mathrm{Pow} \colon \mathrm{Set} \to \mathrm{Set}
%\]
%which associates a set \(S\) to its power set
%\[
%  \mathrm{Pow}(S) := \{ X \mid X \subset S \}
%\]
%and which associates a function \(f \colon S \to T\)
%the inverse image
%\begin{align*}
%  \mathrm{Pow}(f) \colon \mathrm{Pow}(T) &\to \mathrm{Pow}(S) \\
%  Y & \mapsto f^{-1}(Y).
%\end{align*}
%We check this is a contravariant functor...



This is in response to this prompt:

\begin{quote}
Consider the category of sets. Let X and Y be two sets. The Cartesian product
\[
X \times Y = \{(x,y) | x \in X, y \in Y\}
\]
is another set. Moreover this has the following property: given any other set \(Z\) and two functions \(f: Z \to X\) and \(g: Z \to Y\), there is a unique function \(h: Z \to X \times Y\) such that \(h; pr_X = f\) and \(h ; pr_Y = g\). Here, \(pr_X: X \times Y \to X\) and \(pr_Y: X \times Y \to Y\) are the projection maps. (One might say that some diagram commutes…) Compare this with the notion of a categorical product. 

There are two things to check: that there is a map, and that it is unique. It is somehow so tautological that the proof might be hard.
\end{quote}

(Remember we are using the semicolon `followed by' operator \(f; g = g \circ f\) for reversed function composition.)

\subsection{products are products}

When I was a kid, there was a `puzzle' or `paradox', possibly inspired by superhero stories, asking what would happen, if an unstoppable force met an immovable obstacle.\footnote{https://en.wikipedia.org/wiki/Irresistible\_force\_paradox}
I don't remember whether it was my older cousin who told it to me, but it might have been, and let's say it was.
It was a bit cruel (kids are often a bit cruel to each other), or at least, it felt a bit cruel to me.
When my cousin asked the question,
they subtly implied that the answer was either one or the other (as it turns out, a false dichotomy),
and that the question was a matter of taste. It is not a matter of taste. In contrast, ``Who is your favorite superhero?'' is a question that \emph{is} a matter of taste).
Then whichever side I picked, my cousin would talk about the opposite side's guarantee,
building up their specified superpower by using emphasis and superlatives,
saying things like ``but this force is \emph{absolutely unstoppable}'',
until I felt bullied into switching my answer.

For me, the key to eventually understanding what was going on was to stop narrowly trying to work forward from the stipulated qualities, or narrowly trying to work backward from one or the other (presumed) possibly-correct answers, but instead to relax and think about a large grid of "when this bumps that, what happens?", possibly with little xkcd-style illustrations\footnote{something like https://xkcd.com/2043/ or https://xkcd.com/1890/} in each cell.


I think of this as `naturalistic' thinking, your mind having a gentle and wide grip rather than a narrow tight focus. The idea is that we are collecting flower-theorems from this entire field of flower-theorems, all of which are valuable. The naturalist-mathematician is interested in building a catalog of all the flower-theorems, not breaking down anything or building up to anything in particular.

The quality of `unstoppableness' is like circling a row and saying that everywhere in this row, the bumper overcomes the bumpee.
The quality of `irresistablenesss' is like circling one column and saying that everywhere in this column, the bumpee stops the bumper.
The grid and the circling can resolve the paradox.
In this case, the resolution is that the puzzle setter has contradicted themselves.

There's a similar puzzle\footnote{https://en.wikipedia.org/wiki/Teumessian\_fox} regarding a dog, Laelaps, who is guaranteed to succeed in catching everything it hunts and a fox who is guaranteed to evade any pursuer,
which I mention because the resolution to the puzzle is different in this case: the hunt turns out to be nonterminating.

Turing's halting problem and Cantor's proof of the uncountability of real numbers are
more arguments which I think of as flowing from this `think about the whole grid of...'
heuristic.

In group theory, the identity satisfies the law that for all \(x\), \(x * e = e * x = e\).
Suppose that there was another element \(e_2\) which also satisfied a law that for all \(x\), \(x * e_2 = e_2 * x = e_2\).
Satisfying this property means there is a whole row (the \(e_2\) row) and a whole column (the \(e_2\) column) filled with \(e_2\) in the multiplication table of the group.
Of course, there is also a whole row (the \(e_1\) row) and a whole column (the \(e_1\) column) filled with \(e_1\).
What happens when they intersect?

Here again the resolution is different, the four products \(e\{1,2\} * e\{1,2\}\) can all be equal to \(e_1\) and also all be equal to \(e_2\), if \(e_1\) is equal to \(e_2\).
\[
e_1 = e_{\{1,2\}} * e_{\{1,2\}} = e_2
\]

Note that we are using braces and commas here to express a set of alternatives\footnote{https://www.linuxjournal.com/content/bash-brace-expansion}, for example \(\{a,b\} hello \{2,3\}\) expands to the four strings \(a hello 2, a hello 3, b hello 2, b hello 3\).

In category theory, an initial object, 0,  satisfies the law that for all objects \(x\) there exists a unique map \(0 \to x\). What if there were two initial objects, \(0_1\) and \(0_2\)?
Then we could draw a grid of all the \(hom(x, y)\), and look at the rows and columns involving \(0_1\) and \(0_2]\).
The whole row of \(0_1\) is filled with singleton \(hom(0_1, y)\) sets,
and the whole row of \(0_2\) is filled with singleton \(hom(0_2, y)\) sets.
Looking down the columns, we don't know much - except of course where they intersect the rows.

For all four \(0_{\{1,2\}} \to 0_{\{1,2\}}\) there is a unique map,
and we know some \(0_1\to0_1\) and \(0_2\to0_2\) maps, the identity maps, already.
So \(0_1\to 0_2\) composed with \(0_2 \to 0_1\) must yield the identity map \(id 0_1 : 0_1 \to 0_1\),
and \(0_2 \to 0_1\) composed with \(0_1 \to 0_2\) must yield the identity map \(id 0_2 : 0_2 \to 0_2\).
That is, we use the two object's guarantees (their posited superpowers) to obtain an isomorphism between them.

The resolution of the argument regarding cartesian product and categorical product is close to the group-identities-must-be-equal and the initial-objects-must-be-equal
argument - we end up with the cartesian product and categorical product being isomorphic.

During the course of the argument, we need to hold the cartesian product and categorical product as
if they are distinct. But the similarity of the concepts means that its easy to be confused,
both as proof-writer and proof-reader, and (due to the similarity of the concepts) they also have 
similar names, which also makes it confusing.

\begin{definition}
Given \(X, Y\) sets, 
\(
\operatorname{Decartes} X Y := \{(x,y) | x \in X, y \in Y\}
\)
\end{definition}

\begin{theorem}[the universal property of the cartesian product, stated without proof]
Given any other set Z and two functions \(f: Z \to X\) and \(g: Z \to Y\),
there is a unique function \(h: Z \to \operatorname{Decartes} X Y\)
such that \(h; \operatorname{pr} X = f\) and \(h; \operatorname{pr} Y = g\).
Here \(\operatorname{pr} X: \operatorname{Decartes} X Y \to X\) and \(\operatorname{pr} Y: \operatorname{Decartes} X Y \to Y\) are the projection maps.
\end{theorem}


%Definition of Categorical Product (reworded from Wikipedia):
\begin{definition}

Fix a category \(C\). Let \(X_1\) and \(X_2\) be objects of C. A product of \(X_1\) and \(X_2\) is an object \(\operatorname{Cross} X_1 X_2\),
with a pair of arrows \(p_1: \operatorname{Cross} X_1 X_2 \to X_1\), \(p_2: \operatorname{Cross} X_1 X_2 \to X_2\)
such that
for every object Y and every pair of arrows \(f_1: Y \to X_1\), \(f_2: Y \to X_2\),
there exists a unique morphism \(f: Y \to \operatorname{Cross} X_1 X_2\) such that
\(f; p_1 = f_1\)
and 
\(f; p_2 = f_2\)
\end{definition}

\begin{theorem}
The Categorical Product is isomorphic to the Cartesian Product in Set.
\end{theorem}

{\huge beware below is not really latexed}

\begin{proof}

%%%%%%%%%%%%%%%%%%%%%%%%%%%%%%%%%%%
%%% BELOW IS NOT LATEX
%%%%%%%%%%%%%%%%%%%%%%%%%%%%%%%%%%%


We have X, Y sets. We want to show that Decartes X Y is isomorphic to Cross X Y.
We take 1. Decartes X Y,
2. the cartesian projection decartesP 1: Decartes X Y -> X and
3. the cartesian projection decartesP 2: Decartes X Y -> Y
and use them with Cross X Y's guarantee to get
4. a unique arrow dc: Decartes X Y -> Cross X Y
such that
5. dc; crossP 1 = decartesP 1
and
6. dc; crossP 2 = decartesP 2

We take 7. Cross X Y,
8. the cross projection crossP 1: Cross X Y -> X and
9. the cross projection crossP 2: Cross X Y -> Y
and use them with Decartes X Y's guarantee to get
10. a unique arrow cd: Cross X Y -> Decartes X Y
such that
11. cd; decartesP 1 = crossP 1
and
12. cd; decartesP 2 = crossP 2

We take 13. Decartes X Y, decartesP 1 and decartesP 2,
and use them with Decartes X Y's guarantee to get
14. a unique arrow dd: Decartes X Y -> Decartes X Y
such that
15. dd; decartesP 1 = decartesP 1
16. dd; decartesP 2 = decartesP 2

We know another arrow, id (Decartes X Y), which has those properties (15 and 16)
but dd was guaranteed to be unique (14) and so dd equals id.

Consider dc; cd.

17.

dc; cd; decartesP x
= (by cd's guarantee. 11 or 12)
dc; crossP x
= (by dc's guarantee, 5 or 6)
decartesP x

We know another arrow, dd = id which has those properties (17), 
but it was guaranteed to be unique (4)
and so dc; cd = dd = id.

By symmetrical argument, cd; dc = id. Therefore Decartes X Y is isomorphic to Cross X Y.
\end{proof}

\subsection{substituting isomorphic objects leads to equivalent categories}

Before spending time thinking about it, I worried that substituting isomorphic objects for each other
might be `visible' in some sense via arrow composition maybe distant from the isomorphic objects. This is
an argument that we are licensed by an isomorphism to do `surgery' on the set of objects of a category,
collapsing pairs of isomorphic objects into one another.
\marginpar{maybe related to skeletal categories}

Given a category X, and a pair of purportedly distinct objects in it named Decartes and Cross,
and arrows dc: Decartes -> Cross and cd: Cross -> Decartes, 
such that dc; cd = id and cd; dc = id, let's try to make two new categories:
1. X Minus Decartes, which has almost all the objects and arrows in it that X had,
but doesn't have Decartes or any arrows that start or end at Decartes.
2. X Minus Cross, which has almost all the objects and arrows in it that X had,
but doesn't have Cross or any arrows that start or end at Cross.

Can we define a functor from X to X Minus Decartes?
Yes, everything stays where it is, except for the things that got dropped need to be fixed up.
Decartes goes to Cross, and arrows that started or ended at Decartes get pre or post composed with
the distinguished arrows dc and cd.
Let's call this the fixup map from X to X Minus Decartes.

Is fixup a functor? Yes.

\begin{proof}
Suppose we had \(a1; a2 = a3\) in X.

Consider fixup a1; fixup a2.
There are a few cases according to whether a1 or a2 or both had source or target or both equal to Decartes,
but they're all very similar.

We are using brace expansion\footnote{https://www.linuxjournal.com/content/bash-brace-expansion} again.

\begin{align*}
\operatorname{fixup} a1; \operatorname{fixup} a2 & = \{ cd, id \}; a1; \{ cd;dc, id \}; a2; \{ dc, id \} \\     %= (by definition of \operatorname{fixup})
%= (since cd; dc = id)
& = \{ cd, id \}; a1; a2; \{ dc, id \} \\                              
%= (since a1; a2 = a3)
& = \{ cd, id \}; a3; \{ dc, id \} \\                         
%= (by definition of \operatorname{fixup}, reversed)
& = \operatorname{fixup} a3                                                              
\end{align*}

We have shown that a1; a2 = a3 implies fixup a1; fixup a2 = fixup a3. So fixup is a functor.
\end{proof}

The inclusion from X Minus Decartes back to X is also a functor, which we might call iota.

\begin{proof}
If a1; a2 = a3 in X Minus Decartes, then iota a1; iota a2 = iota a3 in X,
because we didn't add or change any arrow compositions,
we only dropped some to create X Minus Decartes.
\end{proof}

Is there a natural transformation from fixup; iota to id?

What does it mean to say there is a natural transformation from fixup; iota to id?
There exists an n: objects of X to arrows of X
such that if you take a generic arrow a of X
then this square commutes:
\[
(\operatorname{fixup}; \iota) a; n (\operatorname{target} a) = n (\operatorname{source} a); (\operatorname{fixup}; \iota) a
\]

Fixup followed by iota slides arrows that might start or finish at Decartes to arrows that instead start or finish at Cross.
So in order for the types to match up, \(n\) should take objects to endo-arrows, except Decartes, which should go to an endo-arrow on Cross.
Let's try defining \(n\) to take everything to its own identity arrow,
except Decartes, which will go to Cross's identity arrow. 

This \(n\) turns out to be a natural transformation from \(\operatorname{fixup}; \iota\) to \(id\).

\begin{proof}
\begin{align*}
(\operatorname{fixup}; \iota) a; n (\operatorname{target} a) & = \iota (\operatorname{fixup} a); n (\operatorname{target} a) \\
& = \iota (\{ 1, cd \}; a; \{ dc, 1 \}); n (\operatorname{target} a) \\      
& = \{ 1, cd \}; a; \{ dc, 1 \}; n (\operatorname{target} a) \\
& = \{ 1, cd \}; a; \{ dc, 1 \}; \{ id \operatorname{Cross}, 1 \} \\
& = \{ 1, cd \}; a; \{ dc, 1 \} \\
& = \{ 1, id \operatorname{Cross} \}; \{ 1, cd \}; a; \{ dc, 1 \} \\
& = \{ 1, id \operatorname{Cross} \}; \iota (\{ 1, cd \}; a; \{ dc, 1 \}) \\
& = \{ 1, id \operatorname{Cross} \}; \iota (\operatorname{fixup} a) \\
& = \{ 1, id \operatorname{Cross} \}; (\operatorname{fixup}; \iota) a \\   
& = n (\operatorname{source} a); (\operatorname{fixup}; \iota) a
\end{align*}
\end{proof}

Is there a natural transformation from iota; fixup to 1? Yes.

\begin{proof}
iota; fixup is an inclusion followed by a function which is the identity on the image of the inclusion,
so it's equal to the identity already.
\end{proof}

Moreover, these natural transformations are actually natural isomorphisms; they always take objects to identity arrows, and identity arrows are always isomorphisms. So X is actually equivalent to X Minus Decartes.

We haven't actually used anything specific about Decartes in these two conclusions, so we also have the analogous maps from C to C Minus Cross and from C Minus Cross back to C.

So X Minus Decartes is equivalent as a category to X Minus Cross; it's not possible to `see' the substitution from a distance via any arrows composing differently.


\end{document}
